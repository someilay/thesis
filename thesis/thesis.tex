\documentclass[oneside,final,14pt,a4paper]{extreport}

\usepackage{tempora} % Times New Roman alike font  
\usepackage{vmargin}
\setpapersize{A4}
\setmarginsrb{2.5cm}{2.2cm}{2.2cm}{2.2cm}{0pt}{10mm}{0pt}{13mm}
\usepackage{setspace}
\setstretch{1.5}
\usepackage{indentfirst}
\parindent=1.25cm

%%%%% ADDED TO SUPPORT TT BOLD FACES %%%%
\DeclareFontShape{OT1}{cmtt}{bx}{n}{<5><6><7><8><9><10><10.95><12><14.4><17.28><20.74><24.88>cmttb10}{}
\renewcommand{\ttdefault}{pcr}
%%%%% END %%%%%%%%%%%%%%%%%%%%%%%%%%%%%%% 

\usepackage{atbegshi,picture}
\usepackage[T1,T2A]{fontenc}
\usepackage{newtxtext}
\usepackage[utf8]{inputenc}

\usepackage[english]{babel}
\usepackage[backend=biber,style=ieee,autocite=inline]{biblatex}
\bibliography{ref.bib}
\DefineBibliographyStrings{english}{%
  bibliography = {References},}
\usepackage{blindtext}

\usepackage{pdfpages}
\newenvironment{bottompar}{\par\vspace*{\fill}}{\clearpage}
\usepackage{amsmath,amsfonts}
\usepackage{url}
\usepackage{amsthm}
\newtheorem{theorem}{Theorem}
\newtheorem{corollary}{Corollary}
\newtheorem{lemma}{Lemma}
\newtheorem{proposition}{Proposition}
\theoremstyle{definition}
\newtheorem{definition}{Definition}
\theoremstyle{remark}
\newtheorem*{remark}{Remark}
\theoremstyle{remark}
\newtheorem*{example}{Example}

\usepackage{float}
\usepackage{graphicx}
\graphicspath{{figs/}} %path to images
\usepackage{array}
\usepackage{multirow,array}
\usepackage{caption}
\usepackage{subcaption}
\usepackage{hyperref}
\hypersetup{colorlinks=true, linkcolor=black, citecolor=black}
\usepackage{paralist}
\usepackage{listings}
\usepackage{zed-csp}
\usepackage{fancyhdr}
\usepackage{csquotes}
\usepackage{color}
% \usepackage{anyfontsize}
% \usepackage{mathptmx}
% \usepackage{t1enc}

\usepackage{chngcntr}
\usepackage{upgreek} 
\usepackage{bm}
\usepackage{hyperref}
\usepackage{booktabs}
\usepackage{multirow}
\usepackage{longtable}
\usepackage[font=singlespacing, labelfont=bf]{caption}
%Hints
\newcommand\pic[1]{(Fig. \ref{#1})} %Ref on figure
\newcommand\tab[1]{(Tab. \ref{#1})} %Ref on table

\setlength{\headheight}{32.0976pt}
\usepackage{enumitem}
\newlist{inlinelist}{enumerate*}{1}
\setlist*[inlinelist,1]{%
  label=(\arabic*),
}

% \setcounter{secnumdepth}{4}
\captionsetup[table]{labelfont={normalfont}, name={TABLE}, labelsep={newline}}
\setlength{\parindent}{2em} 
\DeclareCaptionLabelSeparator{figSep}{.\quad}
\captionsetup[figure]{labelfont={normalfont}, name={Fig.}, labelsep=period}
\counterwithin{figure}{chapter}

% \usepackage{titlesec}
% \titleformat{\section}[hang]{\fontsize{20}{24}\selectfont\filcenter}{\Roman{section}}{1em}{}
% \titleformat{\subsection}[hang]{\itshape}{\Alph{subsection}.}{1em}{}[]
% \titleformat{\subsubsection}[runin]{\itshape}{\arabic{subsubsection})}{1em}{}[$:$]
% \titlespacing{\subsubsection}{1em}{1em}{1em}
% \titleformat{\paragraph}[runin]{\itshape}{\alph{paragraph})}{1em}{}[$:$\quad]
% \titlespacing{\paragraph}{2em}{1em}{1em}

\usepackage{placeins} % for \FloatBarrier
\usepackage{cases}
\usepackage{mathtools}
\usepackage{xcolor}
\usepackage[section]{algorithm}
\usepackage{algpseudocode}
\usepackage{tikz}
\usetikzlibrary{%
    decorations.pathreplacing,%
    decorations.pathmorphing%
}

\pagestyle{fancyplain}

% Remember section title
\renewcommand{\chaptermark}[1]%
	{\markboth{\chaptername~\thechapter~--~#1}{}}

% subsection number and title
\renewcommand{\sectionmark}[1]%
	{\markright{\thesection\ #1}}

\rhead[\fancyplain{}{\bf\leftmark}]%
      {\fancyplain{}{\bf\thepage}}
\lhead[\fancyplain{}{\bf\thepage}]%
      {\fancyplain{}{\bf\rightmark}}
\cfoot{} %bfseries


\newcommand{\dedication}[1]
   {\thispagestyle{empty}
     
   \begin{flushleft}\raggedleft #1\end{flushleft}
}

% Taken from https://cremeronline.com/LaTeX/minimaltikz.pdf
\newcommand{\base}[2]{
  % triangle below circle
  \draw[black, thick] (#1 - 0.3, #2 - 0.5) -- (#1 - 0.1, #2 -0.2);
  \draw[black, thick] (#1 + 0.1, #2 - 0.2) -- (#1 + 0.3, #2 -0.5);

  % base lines
  \draw[black, thick] (#1 - 0.5, #2 - 0.5) -- (#1 + 0.5, #2 - 0.5);
  % dashes
  \draw[black,line width=.5pt,interface](#1 - 0.5, #2 - 0.5)--(#1 + 0.5, #2 - 0.5);
  % circle
  \draw [black, thick,fill=white] (#1, #2) circle [radius=0.2];
}

\newcommand{\drawpointmass}[3][0]{
  % semitransparent circle
  \draw [fill=#1, fill opacity=0.2, draw=#1!40] (#2, #3) circle [radius=0.4];
  % circle
  \draw [black, thick, fill=white] (#2, #3) circle [radius=0.2];
}

\newcommand{\drawpointmasssimple}[3][0.2]{
  % circle
  \draw [black, thick, fill=white] (#2, #3) circle [radius=#1];
}

\newcommand{\drawdot}[3][0.05]{
  % circle
  \draw [black, thick, fill=black] (#2, #3) circle [radius=#1];
}

\newcommand{\gripper}[4]{
  % Adjoint lines
  \draw[black, ultra thick](#1, #2) -- (#1 - #4 / 2 + #2 / 2, #2 + #3 / 2 - #1 / 2);
  \draw[black, ultra thick](#1, #2) -- (#1 + #4 / 2 - #2 / 2, #2 - #3 / 2 + #1 / 2);
  % Parallel lines
  \draw[black, ultra thick]
  (#1 - #4 / 2 + #2 / 2, #2 + #3 / 2 - #1 / 2) -- 
  (#1 - #4 / 2 + #2 / 2 + #3 - #1, #2 + #3 / 2 - #1 / 2 + #4 - #2);
  \draw[black, ultra thick]
  (#1 + #4 / 2 - #2 / 2, #2 - #3 / 2 + #1 / 2) -- 
  (#1 + #4 / 2 - #2 / 2 + #3 - #1, #2 - #3 / 2 + #1 / 2 + #4 - #2);
}

\newcommand{\rectangle}[6]{
  \draw[black, ultra thick](#1, #2) -- (#3, #4);
  \draw[black, ultra thick](#3, #4) -- (#5, #6);
  \draw[black, ultra thick](#1, #2) -- (#1 + #5 - #3, #2 + #6 - #4);
  \draw[black, ultra thick](#1 + #5 - #3, #2 + #6 - #4) -- (#5, #6);
}

\newcommand{\RNum}[1]{\uppercase\expandafter{\romannumeral #1\relax}}

\newcommand{\SE}[1][3]{\text{SE}(#1)}

\newcommand{\SO}[1][3]{\text{SO}(#1)}

\title{Feedback control over constrained robotic systems through the Udvadia-Kalaba approach}
\author{Ilia Milioshin}

\begin{document}
% \includepdf[pages=-]{title.pdf}
\maketitle
\tableofcontents
% \listoftables
% \listoffigures
\newpage

% \begin{abstract}

\par The cooperation of robotic systems, particularly when constrained by shared 
objects, presents a significant challenge in various domains. Previous 
methods, such as models with pre-existing constraints or the 
Karush-Kuhn-Tucker (KKT) approach, suffer from computational complexity and 
prioritization issues. This research solves these limitations by proposing 
a novel methodology based on the Udwadia-Kalaba approach. 
The objective is to enhance control and collaboration of robots using 
advanced tools like MuJoCo, Pinocchio, and ProxSuite. The method integrates 
physical grounding and simplicity, allowing manual fine-tuning of 
constraint priorities and achieving improved computational speed. 
Experimental results demonstrate significant improvements in efficiency, 
accuracy, and adaptability in robotic operations. These advancements suggest 
potential for increased productivity and optimized performance in 
industrial and real-life applications, laying the way for further 
robotics technology developments.

\end{abstract}
\setcounter{page}{7}
% set manually the number, from which Chapter 1 starts!
% Why do we put 7 in this case?
% Title page - page 1
% Contents - page 2, page 3
% List of tables - page 4
% List of figures - page 5
% Abstract - page 6
% Chapter 1 - page 7
% In your thesis the counter number can be different, please count carefully and insert the corresponding number.

\chapter{Introduction}
\label{chap:intro}
% \chaptermark{Optional running chapter heading}

The advancement of microelectronics and sensor technologies has catalyzed a 
widespread integration of robotic systems into various domains. Manipulators, 
delivery robots, and flying drones have become ubiquitous, presenting developers 
and researchers with novel challenges in terms of robustness, adaptiveness, and 
cooperation. Among these challenges, cooperation stands out as a critical 
issue, especially given the prevalence and increasing complexity of such systems. 
In real-world applications, collaboration of robotic systems is crucial across 
various industries. For instance, in manufacturing assembly lines, synchronized 
robots ensure smooth production flow and maximize throughput. Similarly, in 
warehouse logistics, coordinated robotic systems optimize order fulfillment times 
and improve overall productivity. Furthermore, collaborative robotics scenarios in 
construction projects benefit from synchronized robotic systems for tasks like 
concrete pouring and steel beam placement, improving efficiency and minimizing 
delays. A common challenge in this realm involves effectively controlling robots 
constrained by a shared object. This paper proposes a straightforward 
methodology to solve such problems through the 
Udwadia-Kalaba\cite{UdwadiaKalabaApproach} approach.

This proposed methodology gains particular significance within the context 
of modern physics simulation, derivation, and optimization libraries. These 
tools offer substantial computational speed, enabling efficient 
problem-solving. In this article, I relay MuJoCo\cite{MuJoCo}, 
Pinocchio\cite{Pinocchio}, and ProxSuite\cite{ProxQP} 
for simulation, robotic dynamics computation, and optimization, respectively. 
Pinocchio, in particular, appears as a key tool for computing the dynamics of 
robotic systems. However, its limitation to open-loop physics models poses 
challenges for controlling multiple robots with cooperation tasks.

Previous solutions have often involved constructing models with pre-existing 
constraints or utilizing the KKT (Karush-Kuhn-Tucker) approach. 
However, these methods suffer from computational complexity and issues with 
constraint prioritization. The proposed methodology combines the strengths 
of both approaches, integrating physical grounding from the first one and 
simplicity of application from the second one. Notably, it allows for 
manual fine-tuning of constraint priorities and boasts improved 
computational speed through possible auto code generation. 

The implementation of the proposed methodology demonstrates significant 
advancements in the control and collaboration of robotic systems 
constrained by shared objects. Through the integration of robust physics 
simulation, precise robotic dynamics computation, and efficient 
optimization techniques, the method achieves remarkable outcomes by 
simulation experiment. By relying on advanced simulation, computation, 
and optimization techniques, the method has improved levels of efficiency, 
accuracy, and adaptability in robotic operations, laying the way for 
further advancements in robotics technology.

The Atlas robot, developed by Boston Dynamics, serves as an ideal platform 
to demonstrate this methodology's capabilities. In a scenario where 
multiple Atlas robots are tasked with collaboratively manipulating a large, 
heavy object, the shared object constraint requires precise coordination 
and control. Integrating this methodology with Atlas robots can demonstrate improved 
collaborative capabilities. The robust simulation ensures accurate 
interaction modeling, while precise dynamics computation allows effective 
control. Optimization techniques facilitate optimal task prioritization, 
significantly improving overall performance. This application validates 
the methodology and shows its potential for advancing collaborative robotic 
systems.

In subsequent chapters, I deepen into various aspects of the proposed method. 
Chapter \ref{chap:lr} provides an comprehensive review of recent literature, 
highlighting the existing solutions and their limitations. 
Chapter \ref{chap:met} clarifies the methodology underlying the proposed 
approach, offering insights into its theoretical foundations. Implementation 
details and code snippets are presented in Chapter \ref{chap:impl}, 
demonstrating the practical application of the method, and commits the 
overall analysis. Finally, Chapter \ref{chap:conclusion} summarizes findings, 
discusses implications, and outlines directions for future research.

\chapter{Literature Review}
\label{chap:lr}
% \chaptermark{Second Chapter Heading}

\chapter{Methodology}
\label{chap:met}

This chapter will present a detailed elucidation of the principles underpinning the 
proposed methodology. The initial section will offer a concise overview of the 
Udvadia-Kalaba's approach is accompanied by a comprehensive exposition of the physical 
rationales employed in this technique. Subsequently, these principles will be 
harnessed in the development of the intended methodology. Furthermore, this section 
will scrutinize the limitations and distinctive characteristics inherent in the 
proposed methodology.

\section{Essentials of Udvadia-Kalaba approach}

Let $\mathbf{q}$ be a vector of generalized coordinates of some physical system.
Also let agree that $\dim \mathbf{q} = n_q$ and 
$\dim \dot{\mathbf{q}} = \dim \ddot{\mathbf{q}} = n_v$, where $\dot{\mathbf{q}}$ and 
$\ddot{\mathbf{q}}$ are generalized velocity and acceleration respectevly. Notably, 
that in the general case $n_q \neq n_v$ because $\dot{\mathbf{q}}$ and $\ddot{\mathbf{q}}$ 
can be located in different spaces. Thus, the dynamics of the given system can be 
expressed:

\begin{equation}
    \label{eqn:cannonical_man_eq}
    M(\mathbf{q}) \ddot{\mathbf{q}} 
    + C(\mathbf{q}, \dot{\mathbf{q}}) \dot{\mathbf{q}} 
    + g(\mathbf{q}) = \boldsymbol{\tau}
\end{equation}

where $M(\mathbf{q}) \succeq 0 $ is known as general inertia matrix, 
$C(\mathbf{q}, \dot{\mathbf{q}})$ is Coriolis-Centrifugal matrix, and 
$g(\mathbf{q})$ is potential forces impact (usually gravitational impact). 
For further convenience, I would omit the parameters of matrix functions. 
In the case of Coriolis-Centrifugal and potential forces, it is common to 
combine them in one term $\mathbf{Q}(\mathbf{q}, \dot{\mathbf{q}}) = \boldsymbol{\tau}
- C(\mathbf{q}, \dot{\mathbf{q}}) \dot{\mathbf{q}} - g(\mathbf{q})$. Hence, I can 
rewrite \ref{eqn:cannonical_man_eq} in the following manner

\begin{equation}
    \label{eqn:compact_man_eq}
    M \ddot{\mathbf{q}} = \mathbf{Q}
\end{equation}

equation \ref{eqn:compact_man_eq} is compact form of equation 
\ref{eqn:cannonical_man_eq}. It would be highly utilized in further 
investigations.

Let's consider the same physical system again with imposed constraints at this time.
All holonomic constraints would be presented by 
$\varphi(\mathbf{q}, t): \mathbb{S}_{n_q} \times \mathbb{R} \rightarrow \mathbb{R}^m$, and 
all non-holonomic by 
$
\psi(\mathbf{q}, \dot{\mathbf{q}}, t): \mathbb{S}_{n_q} \times \mathbb{R}^{n_v} 
\times \mathbb{R} \rightarrow \mathbb{R}^k
$, where $\mathbb{S}_{n_q}$ generalized coordinates space in the most common case.
All constraints are satisfied if and only if the above functions are equal to zero.

These constraints can be rewritten in the form 
$
A(\mathbf{q}, \dot{\mathbf{q}}, t) \ddot{\mathbf{q}} 
- b(\mathbf{q}, \dot{\mathbf{q}}, t)
$
by differentiation of holonomic constraints twice and non-holonomic once.
According to Udwadia-Kalaba the constrained force that would satisfy can be 
written

\begin{equation}
    \label{eqn:ud_ka_c_forces}
    Q_c = M^{1 / 2} (A M^{-1/2})^+(b - A M ^ {-1} Q)
\end{equation}

where $M^{\pm 1 / 2} = W \Lambda^{\pm 1/ 2} W^T$ ($W, \Lambda$ gain by 
eigendecomposition), $[*]^+$ is the Moore-Penrose inverse. Under these forces, 
the solution of the forward dynamics takes the form

\begin{equation}
    \label{eqn:forward_dyn_with_cf}
    \ddot{\mathbf{q}} = M^{-1}Q + M^{-1 / 2} (A M^{-1/2})^+(b - A M ^ {-1} Q)
\end{equation}

The equation \ref{eqn:ud_ka_c_forces} according to F. Udvadia and R. Kalaba
\cite{UdwadiaKalabaApproach} is the analytical solution to the minimization problem
based on Gauss's principle of least constraint

\begin{equation}
    \label{eqn:gauss_min_problem}
    \begin{aligned}
        \min_{\ddot{\mathbf{q}}} \quad &
        [\ddot{\mathbf{q}} - a]^T M [\ddot{\mathbf{q}} - a] \\
        \textrm{s.t.} \quad &
        A \ddot{\mathbf{q}} - b = 0 \\
        \quad &
        a(\mathbf{q}, \dot{\mathbf{q}}, t) = M^{-1} Q
    \end{aligned}
\end{equation}

While the solution proposed by F. Udvadia and R. Kalaba demonstrates precision, 
it is not without its drawbacks. Instability points and computationally intensive 
operations, such as the computation of matrix roots, present significant challenges. 
Moreover, a notable deficiency lies in the absence of prioritization within the 
methodology.

\chapter{Implementation and evaluation}
\label{chap:impl}

In this chapter the implementation of proposed method and results 
of numerical experiments are shown. Firstly, Section {\ref{sec:dual_arm_yumi}} 
contains information about chosen system and its properties. Then, in Section 
\ref{sec:frameworks} the discussion of used frameworks and utils is presented. 
Consequently, these tools are used in Section \ref{sec:impl_details} to show 
how the proposed algorithms can be implemented. Finally, Section 
\ref{sec:sim_results} contains results of numerical experiments with their 
evaluation.

\section{Dual-arm YuMi}
\label{sec:dual_arm_yumi}

To prove the workability of the suggested methodology the dual-arm YuMi 
manipulator is chosen, see Fig. \ref{fig:dual_arm_yumi}. It has 18 DoF, namely: 
7 rotation and 2 prismatic (fingers) joints per each arm. 

In simulation the total DoF was reduced to 14 by fixing end-effectors' fingers. 
Also for demonstration purposes the joint limits and body collisions were 
removed. Otherwise, their presence could influence on final results.

\begin{figure}[H]
    \centering
    \includegraphics[scale=0.25]{figs/yumi.png}
    \caption{Dual-arm YuMi IRB 14000}
    \label{fig:dual_arm_yumi}
\end{figure}

The discussed manipulator was selected because there is URDF (Unified Robot 
Description Format) for it. However, there was no necessary representation for the 
chosen simulator. Thus, it was generated. Details is conducted later 
in Section \ref{sec:impl_details}.

\section{Frameworks}
\label{sec:frameworks}

For implementation the discussed algorithms C++ is used. Beyond of chosen language 
it is also necessary to select a simulator, a tool for computing dynamics, and 
library for solving the Gauss least constraint principle 
(\ref{eqn:least_act_principle_const}). In this section all of them is reviewed.

Firstly, let's discuss chosen simulator, MuJoCo \cite{MuJoCo}. It is general 
physical engine. In this study this framework is used to calculate system 
behavior under given Control. Also, it provides way to visualize robot. From 
control point of view MuJoCo is only needed to give $\mathbf{q}$ and $\mathbf{v}$ 
at each step. 

The system state is transferred to dynamic computing tool. This paper is utilizing 
Pinocchio \cite{Pinocchio} library. It is modern and efficient instrument for 
calculating kinematic and dynamic for given system. Notably, this frameworks 
supports the URDF, which is very convenient in the context of the selected system. 
Moreover, Pinocchio provides fast computational utils. Authors of tool give 
the following quantities, see Fig. \ref{fig:pin_speed}. For implementation this 
library is necessary to calculate Jacobians, attitudes of frames, and velocities.

\begin{figure}[H]
    \centering
    \includegraphics[scale=0.2]{figs/pin_speed.png}
    \caption{Pinocchio performance demonstration}
    \label{fig:pin_speed}
\end{figure}

Finally, the last needed element is optimization problem solver. In this paper 
ProxSuite \cite{ProxQP} is used for such purpose. The suggested methodology lies 
on quadratic problem. Authors of ProxSuite shows that their tool is fast of 
solving it. The mean computation time is 10 microseconds, see Fig. 1 in 
\cite{ProxQP}. Although, the discussed instrument is efficient but the problem 
(\ref{eqn:gauss_least_const_prior}) can be directly solved. Thus, it is needed 
to be reformulated. Further details is provided later in Section \ref{sec:impl_details}

\section{Implementation details}
\label{sec:impl_details}

This section contains details of implementation of the suggested methodology. It 
can be split into two parts, namely: computing $A$ and $\mathbf{b}$ with Pinocchio, 
and solving the problem (\ref{eqn:gauss_least_const_prior}) by ProxSuite. 
Remarkably, in this section there is no details about simulation and visualization 
because they described in MuJoCo documentation. However, it is necessary to 
mention that there is no YuMi representation for this simulator. Let's 
first discuss this moment.

The MuJoCo physical engine supports only specific XML description. It is usually 
called MJCF. For numerical experiment the total DoF was reduced to 14. Only rotation 
joints were remained. For demonstration purposes it is enough.

Now let's consider implementation of proposed algorithms 
\ref{alg:get_a_and_b}, \ref{alg:get_a_and_b_control_fixed}, and 
\ref{alg:get_a_and_b_control_rigid}. For all of them it is obligatory to compute 
Jacobians, attitude of frames, and velocities. Let's starts from system model 
and data definition:

\begin{lstlisting}[caption={Model and data}, label=snp:model_and_data]
namespace pin = pinocchio;
...
pin::Model m;
pin::urdf::buildModel("path/to/urdf", m);
pin::Data d(m);
\end{lstlisting}

Next it is necessary to compute forward kinematics for given model. As input this 
step requires generalized coordinates and velocities ($\mathbf{q}$ and 
$\mathbf{v}$). See below snippet:

\begin{lstlisting}[caption={Forward kinematics}, label=snp:forward_kin]
pin::forwardKinematics(m, d, q, v);
pin::computeJointJacobians(m, d, q);
pin::computeJointJacobiansTimeVariation(m, d, q, v);
pin::updateFramePlacements(m, d);
\end{lstlisting}

Now Jacobians, attitudes of frames, and velocities can be calculated. If $E$ is 
origin of some frame attached to system, then it has a special id in Pinocchio 
description. Let's call it \texttt{eIdx}. Thus, the code for Jacobians is

\begin{lstlisting}[caption={Jacobians}, label=snp:jacs]
using MatXd = Eigen::MatrixXd;
...
MatXd jac = MatXd::Zero(6, m.nv);
MatXd dJac = MatXd::Zero(6, m.nv);
pin::getFrameJacobian(m, d, eIdx, pin::LOCAL_WORLD_ALIGNED, jac);
pin::getFrameJacobianTimeVariation(m, d, eIdx, pin::LOCAL_WORLD_ALIGNED, dJac);
\end{lstlisting}

Here \texttt{jac} and \texttt{dJac} are Jacobian and its derivative over time 
respectively. It is important to mention that these quantities from 
$\mathbb{R}^{6 \times n_v}$. The first three rows are corresponding to 
classical linear velocity, $J_{E,v}$ or $\dot{J}_{E,v}$. The bottom three rows are 
angular part, $J_{E,\omega}$ or $\dot{J}_{E,\omega}$. All of them are expressed 
in the world coordinates.

For attitudes and velocities it is enough to execute

\begin{lstlisting}[caption={Attitudes of frames}, label=snp:atts_of_frames]
pin::Frame eFrame = m.frames[idx];
pin::FrameIndex pIdx = frame.parent;
pin::SE3 localToWorldY = pin::SE3::Identity();
localToWorldY.rotation(d.oMf[eIdx].rotation());
pin::SE3 eAtt = d.oMf[eIdx]; 
pin::Motion eVel = localToWorldY.act(frame.placement.actInv(d.v[pIdx]));
\end{lstlisting}

In snippet (\ref{snp:atts_of_frames}) \texttt{eAtt} and \texttt{eVel} are 
frame attitude and velocity respectively. \texttt{eVel} contains linear and 
angular part.

The last parts worth mentioning are the skew-symmetric operator ($[.]^{\wedge}$) 
and matrix logarithm ($\log (.)$). In Pinocchio they are \texttt{pin::skew} and 
\texttt{pin::log3} function respectively. Other mathematical operations are not 
worth to be explained due to their triviality. 

All snippets above are used for experiment. They can be found in the repository 
\cite{experimentsRepo}. However, for optimization and readability they 
are realized not in the same order. In the provided code algorithm 
\ref{alg:get_a_and_b} is demonstrated in \texttt{getConstraintsAffineDesc}, 
algorithm \ref{alg:get_a_and_b_control_fixed} is shown in \\
\texttt{getControlAffineDesc}, the last one \ref{alg:get_a_and_b_control_rigid} 
can be implemented by function \\ \texttt{getAffineDesc}.

Now let's discuss the transformation of equation 
(\ref{eqn:gauss_least_const_prior}) for ProxSuite framework. This library can 
handle quadratic problem of the following form

\begin{equation}
    \begin{aligned}
        \min_{\mathbf{x}} \quad & 
        \frac{1}{2} \mathbf{x}^T H \mathbf{x} + 
        \mathbf{g}^T \mathbf{x} \\
        \textrm{s.t.} \quad & A \mathbf{x} = \mathbf{b} \\
        & \mathbf{y}_l \leq C \mathbf{x} \leq \mathbf{y}_u  
    \end{aligned}
    \label{eqn:prox_qp_form}
\end{equation}

where $[.] \leq [.]$ is element-wise vector operator, $H \succeq 0$, and 
other quantities are arbitrary.

In case of the Gauss least constraint principle the last line of quadratic 
problem (\ref{eqn:prox_qp_form}) is unnecessary. Hence, it is enough 
to pen brackets of equation (\ref{eqn:gauss_least_const_prior}):

\begin{equation}
    \begin{aligned}
        \min_{\dot{\mathbf{v}}} \quad & 
        \frac{1}{2} \dot{\mathbf{v}}^T [M + \omega A_u^T A_u] \dot{\mathbf{v}} + 
        [-\mathbf{Q} - \omega A_u^T \mathbf{b}_u]^T \dot{\mathbf{v}} \\
        \textrm{s.t.} \quad & A_c \dot{\mathbf{v}} = \mathbf{b}_c
    \end{aligned}
    \label{eqn:gauss_least_const_to_prox_qp}
\end{equation}

The removing of unconstrained acceleration $\mathbf{a}$ is possible by equation 
(\ref{eqn:can_man_eqn_simple}). Thus, $H = M + \omega A_u^T A_u$ and 
$\mathbf{g} = -\mathbf{Q} - \omega A_u^T \mathbf{b}_u$. The problem 
(\ref{eqn:gauss_least_const_to_prox_qp}) is solvable by the following 
code

\begin{lstlisting}[caption={Quadratic problem solution}, label=snp:qp_sol]
namespace prox = proxsuite::proxqp;
...
prox::dense::QP<double> qp(m.nv, 6, 0);
qp.settings.eps_abs = eps_abs;
qp.settings.initial_guess = prox::InitialGuessStatus::NO_INITIAL_GUESS;
qp.settings.verbose = false;
qp.init(
    M + omega * Au.transpose() * Au,
    -Q - omega * Au.transpose() * bu,
    Ac, bc,
    proxsuite::nullopt, proxsuite::nullopt, proxsuite::nullopt
);
qp.solve();
\end{lstlisting}

After first computation it is recommended to use \\
\texttt{prox::InitialGuessStatus::WARM\_START} to speed up calculation. This 
flag enables initial guess taken from previous solution.

\section{Simulation results}
\label{sec:sim_results}

During the experiment the following presets for YuMi are following

\begin{longtable}{|p{.30\textwidth}|l|}
\hline
Quantity & Used in experiment \\ 
\hline
Frame $\mathcal{E}_1$ & Joint frame \texttt{yumi\_link\_7\_l} in URDF \\ 
\hdashline
Frame $\mathcal{E}_2$ & Joint frame \texttt{yumi\_link\_7\_r} in URDF \\ 
\hdashline
$T_{\mathcal{A}}^{\mathcal{B}}$ & 
$\begin{bmatrix}
    1 & 0 & 0 & 0 \\
    0 & 0 & 1 & -0.4 \\
    0 & -1 & 0 & 0.12 \\
    0 & 0 & 0 & 1  
\end{bmatrix}$ \\
\hdashline
Constraint $K_p$ & 1000 \\
\hdashline
Constraint $K_d$ & 33 \\ 
% \hline
Control $K_p$ & 500 \\ 
\hdashline
Control $K_d$ & 17 \\ 
\hdashline
$\omega$ from equation (\ref{eqn:gauss_least_const_prior}) & 50 \\ 
\hdashline
Desired law ($T_{\mathcal{W}}^{\mathcal{D}}$) of motion for $\mathcal{E}_1$ & 
$\begin{bmatrix}
    1 & 0 & 0 & 0.436763 + 0.12 \cdot \cos(1.5 t + \pi / 2) \\
    0 & 1 & 0 & 0.185684 \\
    0 & 0 & 1 & 0.512661 + 0.12 \cdot \sin(1.5 t + \pi / 2) \\
    0 & 0 & 0 & 1
\end{bmatrix}$ \\ 
\hdashline
Initial $\mathbf{q}$ & See in \cite{experimentsRepo} \\ 
\hdashline
Initial $\mathbf{v}$ & $\mathbf{0}$ \\ 
\hline
\end{longtable}

The initial $\mathbf{q}$ is around of ideal one that gives zero control and 
constraint error. Simulation shows the following behavior of generalized 
coordinates and velocities under proposed control, see Fig. \ref{fig:q_and_v_plot}

\begin{figure}[H]
    \centering
    \includegraphics[scale=0.52]{figs/q_and_v_history.png}
    \caption{Experimental $\mathbf{q}$ and $\mathbf{v}$}
    \label{fig:q_and_v_plot}
\end{figure}

The plot of $\|\mathbf{q}\|$ and $\|\mathbf{v}\|$ over time demonstrates 
convergence to some periodical function. It can be explained by chosen 
desired law with period $4 \pi / 3$. Although, it is indirect proof of success, but 
the plot of positions shows stability of proposed methodology, see Fig. 
\ref{fig:poses_plot}. Subscript $s$ stands for $\mathcal{E}_1$ frame, 
and $e$ stands for $\mathcal{E}_2$.

\begin{figure}[H]
    \centering
    \includegraphics[scale=0.52]{figs/poses_history.png}
    \caption{Experimental frame positions}
    \label{fig:poses_plot}
\end{figure}

\begin{figure}[H]
    \centering
    \includegraphics[scale=0.52]{figs/vels_history.png}
    \caption{Experimental frame velocities}
    \label{fig:vels_plot}
\end{figure}

The graph of difference between desired position and actual one converges to 
zero with some numerical noise. In Fig. \ref{fig:poses_plot} the obtained 
values are mean after $t > 1.5$. The steady state response time is less 
than $0.75$ second according to Fig. \ref{fig:vels_plot}. Futhermore, 
the $\mathbf{e}_{\SE}$ for constraint converges faster, see  
Fig. \ref{fig:errors_plot}. However, control error converges to zero 
same as velocities.

\begin{figure}[H]
    \centering
    \includegraphics[scale=0.52]{figs/errors_history.png}
    \caption{Experimental $\mathbf{e}_{\SE}$, left $\mathbf{e}_c$ - constraint, 
    right $\mathbf{e}_u$ - control}
    \label{fig:errors_plot}
\end{figure}

The obtained results of numerical experiment proves the workability 
of suggested methodology. Nevertheless, results demonstrate convergence 
of over-damped system. Thus, finetuning of $K_p$, $K_d$ is still needs 
further investigations.

\chapter{Conclusion}
\label{chap:conclusion}

This research developed a straightforward methodology for 
coupling robotic systems constrained by a shared object using the 
Udwadia-Kalaba approach. This approach aimed to resolve the challenges of 
robustness, adaptiveness, and collaboration in various real-world applications of 
robotic systems. This study revealed that the proposed methodology achieves the 
stable control and interaction of robotic systems. These results are obtained 
by integrating modern physics simulation, robotic dynamics computation, and 
optimization libraries such as MuJoCo, Pinocchio, and ProxSuite. 
Specifically, the methodology demonstrated improved computational speed with 
keeping high precision in robotic operations. It allowed for manual fine-tuning of 
constraint priorities, facilitating flexibility and control over robotic 
collaboration. Through numerical experiments proved the methodology enhanced 
performance in coupling robots in collaborative work with solid objects. 
Especially, the described approach showed simplicity of creating and manipulating 
constraints.

This research provides a robust solution to the interacting complex 
robotic systems, which is critical for improving productivity and efficiency in 
various industrial and real-life applications. The integration of advanced 
computational tools enables more effective problem-solving, that consequently 
advances the capabilities of automated and robotic systems. The proposed 
methodology offers a novel combination of the Udwadia-Kalaba approach 
with modern simulation and optimization tools. It overcomes the limitations of 
previous methods such as the complexity and constraint prioritization issues. 
Futhermore, the suggested method is a practical and efficient solution 
for real-time interaction of multiple robots constrained by shared objects.

Despite its contributions, the research has several limitations. The study assumes 
systems of rigid bodies with at most inner stiffness, which may not fully 
represent all real-world robotic applications. Also, the approach requires a 
predefined description of the entire system, limiting its applicability in 
scenarios with changing or unknown system dynamics. Future research should 
focus on extending the methodology to handle non-rigid bodies and more complex 
physical interactions. Also, further studies should investigate finetuning $K_p$, 
$K_d$ parameters. These research should be aimed to two purposes, namely: 
enhancing the method's ability to work with wider range of physical systems, and 
conducting more extensive real-world experiments to validate 
the methodology's effectiveness. By addressing these areas, future research can 
further expand the applicability of this collaboration 
approach, laying the way for more advanced and efficient robotic systems.

\include{chapters/chapter6}

\appendix
\chapter{Brief Lie theory}

The Lie group, a mathematical concept dating back to the 19th century, was 
first proposed by Sophus Lie, laying the foundation for continuous 
transformation groups. Although it was initially abstract, over time its 
influence has spread to various scientific and technological fields. The 
mathematical tools discussed in this chapter are crucial for the research 
discussed here. It inspired by this study \cite{MicroLieTheory}, and 
contains only crucial information.

\section{Groups, Algebras, and operators}
\label{sec:group_algebra_ops}

Let $\mathcal{G}$ is smooth manifold that satisfies the group axioms. For any 
$\mathcal{X}$, $\mathcal{Y}$ and $\mathcal{Z}$ from $\mathcal{G}$ the following 
statements are true:

\begin{equation}
    \begin{aligned}
        \label{eqn:g_axi}
        & \mathcal{X} \circ \mathcal{Y} \in \mathcal{G} \quad && \text{(I)} \\
        & \exists \mathcal{E}:\ \mathcal{E} \circ \mathcal{X} 
        = \mathcal{X} \circ \mathcal{E} = \mathcal{X} \quad && \text{(II)} \\
        & \exists \mathcal{X}^{-1}:\ \mathcal{X}^{-1} \circ \mathcal{X}
        = \mathcal{X} \circ \mathcal{X}^{-1} = \mathcal{E} \quad && \text{(III)} \\ 
        & (\mathcal{X} \circ \mathcal{Y}) \circ \mathcal{Z} = 
        \mathcal{X} \circ (\mathcal{Y} \circ \mathcal{Z}) \quad && \text{(IV)}
    \end{aligned}
\end{equation}

For such group $T_{\mathcal{E}} \mathcal{G}$ is the Lia Algebra defined at 
$\mathcal{E}$ element. The geometric interpretation of algebra is tangent plane 
that touches the smooth manifold (group). $\mathfrak{m} \equiv 
T_{\mathcal{E}} \mathcal{G}$ is always a vector space.

The next crucial definition is a group action

\begin{equation}
    f: \mathcal{G} \times \mathcal{V} \to \mathcal{V}
\end{equation}

where $\mathcal{V}$ is some set. The operation defined above should satisfy 
the axioms ($v \in \mathcal{V}$; $\mathcal{X}, \mathcal{Y} \in \mathcal{G}$),

\begin{equation}
    \begin{aligned}
        \label{eqn:act_axi}
        & \mathcal{E} \cdot v = v \quad && \text{(I)} \\
        & (\mathcal{X} \circ \mathcal{Y}) \cdot v = 
        \mathcal{X} \cdot (\mathcal{Y} \cdot v) \quad && \text{(II)}
    \end{aligned}
\end{equation}

For instance, if the $\text{SO}(n)$ rotations is considered as Lie group, than the 
transformation $R \cdot x \equiv R x$ ($x \in \mathbb{R}^n$) is a group action.

The exponential map $\exp:\ \mathfrak{m} \to \mathcal{G}$ converts elements of 
Lie algebra to corresponding group. The log map do inverse operation. 
However, it is necessary to remember that 
this map applicable only for "identity" algebra. However, there is a linear
transformation between $T_{\mathcal{X}} \mathcal{G}$ and $T_{\mathcal{E}} 
\mathcal{G}$. It is called adjoin.

It is known that $\mathfrak{m}$ is isomorphic to the vector space $\mathbb{R}^m$. 
One can write it as $\mathfrak{m} \cong \mathbb{R}^m$. Due to convenience this 
isomorphism would be highly utilized. Moreover, in the bellow sections only 
algebras constructed on $\mathbb{R}^m$ are considered. Thus, a mapping between 
sets can be defined in the following manner (hat-vee notation)

\begin{equation}
    \begin{aligned}
        \label{eqn:hat_vee_not}
        & [*]^\wedge: \ \mathbb{R}^m \to \mathfrak{m} \quad &&
        \boldsymbol{\tau}^\wedge = \sum_{i=1}^{m} \tau_i E_i \\
        & [*]^\vee: \mathfrak{m} \to \mathbb{R}^m \quad && 
        \boldsymbol{\tau} = \sum_{i=1}^{m} \tau_i \mathbf{e}_i
    \end{aligned}
\end{equation}

where $\mathbf{e}_i$ are a base of $\mathbb{R}^m$ and $E_i$ are base vectors 
of $\mathfrak{m}$. Obviously, $\mathbf{e}_i^\wedge = E_i$. Using this mapping, the
exponential / log map can be modified 

\begin{equation}
    \begin{aligned}
        \label{eqn:exp_log_for_r_m}
        & \exp : \quad && \mathcal{X} = \exp(\boldsymbol{\tau}^\wedge) \\
        & \log : \quad && \boldsymbol{\tau} = \log(\mathcal{X})^\vee
    \end{aligned}
\end{equation}

Or, in the most convenient form

\begin{equation}
    \begin{aligned}
        \label{eqn:big_exp_log_for_r_m}
        & \text{Exp} : \quad && \mathcal{X} 
        = \exp(\boldsymbol{\tau}^\wedge) 
        = \text{Exp}(\boldsymbol{\tau})\\
        & \text{Log} : \quad && \boldsymbol{\tau} 
        = \log(\mathcal{X})^\vee
        = \text{Log}(\mathcal{X})
    \end{aligned}
\end{equation}

Through this definitions it easy to introduce plus and minus operations

\begin{equation}
    \begin{aligned}
        & \text{right-} \oplus: &&
        \mathcal{X} \oplus \prescript{\mathcal{X}}{}{\boldsymbol{\tau}} 
        \equiv \mathcal{X} \circ 
        \text{Exp}(\prescript{\mathcal{X}}{}{\boldsymbol{\tau}}) \in 
        \mathcal{G} \\
        & \text{right-} \ominus: &&
        \mathcal{Y} \ominus \mathcal{X} 
        \equiv \text{Log}(\mathcal{X}^{-1} \circ \mathcal{Y}) 
        \in T_{\mathcal{X}}\mathcal{G} \\
        & \text{left-} \oplus: &&
        \prescript{\mathcal{E}}{}{\boldsymbol{\tau}} \oplus \mathcal{X}
        \equiv \text{Exp}(\prescript{\mathcal{E}}{}{\boldsymbol{\tau}}) 
        \circ \mathcal{X} \in \mathcal{G} \\ 
        & \text{left-} \ominus: &&
        \mathcal{Y} \ominus \mathcal{X} \equiv
        \text{Log}(\mathcal{Y} \circ \mathcal{X}^{-1}) 
        \in T_{\mathcal{E}}\mathcal{G}
    \end{aligned}
\end{equation}

Using the above statements it is possible to define Jacobians

\begin{equation}
    \begin{aligned}
        & J_r(\boldsymbol{\tau}): &&
        \text{Exp}(\boldsymbol{\tau} + \delta \boldsymbol{\tau})
        \approx
        \text{Exp}(\boldsymbol{\tau}) \oplus 
        [J_r(\boldsymbol{\tau}) \delta \boldsymbol{\tau}] \\ 
        & J_r^{-1}(\boldsymbol{\tau}): &&
        \text{Log}(\mathcal{X} \oplus \delta \boldsymbol{\tau}) 
        \approx
        \text{Log}(\mathcal{X}) + 
        J_r^{-1}(\boldsymbol{\tau}) \delta \boldsymbol{\tau}
    \end{aligned}
\end{equation}

where $\mathcal{X} = \text{Exp}(\boldsymbol{\tau})$, and $\oplus$, $\ominus$ 
are right versions. The above definitions describe right version of 
Jacobians. There are also left ones that based on left minus and plus 
operators.

\section{Special orthogonal group}

Let's consider $\SO$ (Special orthogonal group). It consists from orthogonal 
rotations matrices, and forms a group under multiplication. According to 
Euler's rotation theorem any orientation can be described by rotation 
about some single axis with precision to $2 \pi n$. Thus, $\SO$ can be 
considered as Lie group, and $\mathbb{R}^3$ is a Lie algebra respectively.

Now let's define a wedge operator $[.]^{\wedge}$:

\begin{equation}
    \hat{\mathbf{a}} = \mathbf{a}^{\wedge} = 
    \begin{bmatrix}
        0 & -a_3 & a_2 \\
        a_3 & 0 & -a_1 \\
        -a_2 & a_1 & 0 \\
    \end{bmatrix}
    \label{eqn:wedge_op}
\end{equation}

where $\mathbf{a} \in \mathbb{R}^3$, and vee operator ($[.]^{\vee}$) is 
following

\begin{equation}
    A^{\vee} = \mathbf{a}, \ \text{where }
    \hat{\mathbf{a}} = A
    \label{eqn:vee_op}
\end{equation}

Using these definitions it is possible to connect $\dot{R}$ with classical 
angular velocity

\begin{equation}
    \begin{aligned}
        \hat{\boldsymbol{\omega}}^s = \dot{R} R^T & \quad
        \hat{\boldsymbol{\omega}}^b = R^T \dot{R}
    \end{aligned}
    \label{eqn:ang_with_dot_r}
\end{equation}

where $b$ and $s$ subscripts denote body and spatial velocities 
respectively.

The exponential map is

\begin{equation}
    R = \text{Exp}(\boldsymbol{\theta}) \equiv
    I + \frac{\sin \theta}{\theta} \hat{\boldsymbol{\theta}} +
    \frac{1 - \cos \theta}{\theta^2} \hat{\boldsymbol{\theta}}^2
    \label{eqn:exp_map}
\end{equation}

where $\boldsymbol{\theta} = \theta \mathbf{u}$, and $\mathbf{u}$ is 
arbitrary unit vector from $\mathbb{R}^3$. Therefore, the logarithm map is

\begin{equation}
    \boldsymbol{\theta} = \text{Log}(R) \equiv 
    \frac{\theta (R - R^T)^{\vee}}{2 \sin \theta}, \ 
    \text{where } 
    \theta = \cos^{-1} \frac{\text{trace}(R) - 1}{2}
    \label{eqn:log_map}
\end{equation}

In the above statement $R - R^T$ is skew symmetric, it can be proved by 
transposing it. 

Now let's move to left and right Jacobians definitions 

\begin{equation}
    \begin{aligned}
        J_r(\boldsymbol{\theta}) & = 
        I - \frac{1 - \cos \theta}{\theta^2} \hat{\boldsymbol{\theta}} +
        \frac{\theta - \sin \theta}{\theta^3} \hat{\boldsymbol{\theta}}^2 \\
        J_r^{-1}(\boldsymbol{\theta}) & = 
        I + \frac{1}{2} \hat{\boldsymbol{\theta}} + 
        \biggl( \frac{1}{\theta^2} - 
        \frac{1 + \cos \theta}{2 \theta \sin \theta} \biggr)
        \hat{\boldsymbol{\theta}}^2 \\ 
        J_l(\boldsymbol{\theta}) & = J_r^T(\boldsymbol{\theta}) \\
        J_l^{-1}(\boldsymbol{\theta}) & = J_r^{-T}(\boldsymbol{\theta})
    \end{aligned}
    \label{eqn:l_and_r_jacobians}
\end{equation}

The all of the above definitions are taken from Appendix B in 
\cite{MicroLieTheory}. The mathematical ground for them can be found 
in \cite{LieStochModels}.


%% REFERENCES
\printbibliography[heading=bibintoc,title={Bibliography cited}]
\end{document}

\chapter{Methodology}
\label{chap:met}

This chapter will provide a detailed explanation of the principles that stay
behind the proposed methodology. In the first half a brief review of Udvadia-Kalaba
approach would be given. Also, this part contains comprehensive physical reasons
that are used in the mentioned technique. In the second half, these principles
will be used to construct the desired methodology. Besides that, the limitations 
and peculiar properties are discussed in this half.

\section{Essentials of Udvadia-Kalaba approach}

Let $\mathbf{q}$ be a vector of generalized coordinates of some physical system.
Also let agree that $\dim \mathbf{q} = n_q$ and 
$\dim \dot{\mathbf{q}} = \dim \ddot{\mathbf{q}} = n_v$, where $\dot{\mathbf{q}}$ and 
$\ddot{\mathbf{q}}$ are generalized velocity and acceleration respectevly. Notably, 
that in the general case $n_q \neq n_v$ because $\dot{\mathbf{q}}$ and $\ddot{\mathbf{q}}$ 
can be located in different spaces. Thus, the dynamics of the given system can be 
expressed:

\begin{equation}
    \label{eqn:cannonical_man_eq}
    M(\mathbf{q}) \ddot{\mathbf{q}} 
    + C(\mathbf{q}, \dot{\mathbf{q}}) \dot{\mathbf{q}} 
    + g(\mathbf{q}) = \boldsymbol{\tau}
\end{equation}

where $M(\mathbf{q}) \succeq 0 $ is known as general inertia matrix, 
$C(\mathbf{q}, \dot{\mathbf{q}})$ is Coriolis-Centrifugal matrix, and 
$g(\mathbf{q})$ is potential forces impact (usually gravitational impact). 
For further convenience, I would omit the parameters of matrix functions. 
In the case of Coriolis-Centrifugal and potential forces, it is common to 
combine them in one term $\mathbf{Q}(\mathbf{q}, \dot{\mathbf{q}}) = \boldsymbol{\tau}
- C(\mathbf{q}, \dot{\mathbf{q}}) \dot{\mathbf{q}} - g(\mathbf{q})$. Hence, I can 
rewrite \ref{eqn:cannonical_man_eq} in the following manner

\begin{equation}
    \label{eqn:compact_man_eq}
    M \ddot{\mathbf{q}} = \mathbf{Q}
\end{equation}

equation \ref{eqn:compact_man_eq} is compact form of equation 
\ref{eqn:cannonical_man_eq}. It would be highly utilized in further 
investigations.

Let's consider the same physical system again with imposed constraints at this time.
All holonomic constraints would be presented by 
$\varphi(\mathbf{q}, t): \mathbb{S}_{n_q} \times \mathbb{R} \rightarrow \mathbb{R}^m$, and 
all non-holonomic by 
$
\psi(\mathbf{q}, \dot{\mathbf{q}}, t): \mathbb{S}_{n_q} \times \mathbb{R}^{n_v} 
\times \mathbb{R} \rightarrow \mathbb{R}^k
$, where $\mathbb{S}_{n_q}$ generalized coordinates space in the most common case.
All constraints are satisfied if and only if the above functions are equal to zero.

These constraints can be rewritten in the form 
$
A(\mathbf{q}, \dot{\mathbf{q}}, t) \ddot{\mathbf{q}} 
- b(\mathbf{q}, \dot{\mathbf{q}}, t)
$
by differentiation of holonomic constraints twice and non-holonomic once.
According to Udwadia-Kalaba the constrained force that would satisfy can be 
written



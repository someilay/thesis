\chapter{Methodology}
\label{chap:met}

This chapter will present a detailed elucidation of the principles underpinning the 
proposed methodology. The initial section will offer a concise overview of the 
Udvadia-Kalaba's approach is accompanied by a comprehensive exposition of the physical 
rationales employed in this technique. Subsequently, these principles will be 
harnessed in the development of the intended methodology. Furthermore, this section 
will scrutinize the limitations and distinctive characteristics inherent in the 
proposed methodology.

\section{Essentials of Udvadia-Kalaba approach}

Let $\mathbf{q}$ be a vector of generalized coordinates of some physical system.
Also let agree that $\dim \mathbf{q} = n_q$ and 
$\dim \dot{\mathbf{q}} = \dim \ddot{\mathbf{q}} = n_v$, where $\dot{\mathbf{q}}$ and 
$\ddot{\mathbf{q}}$ are generalized velocity and acceleration respectevly. Notably, 
that in the general case $n_q \neq n_v$ because $\dot{\mathbf{q}}$ and $\ddot{\mathbf{q}}$ 
can be located in different spaces. Thus, the dynamics of the given system can be 
expressed:

\begin{equation}
    \label{eqn:cannonical_man_eq}
    M(\mathbf{q}) \ddot{\mathbf{q}} 
    + C(\mathbf{q}, \dot{\mathbf{q}}) \dot{\mathbf{q}} 
    + g(\mathbf{q}) = \boldsymbol{\tau}
\end{equation}

where $M(\mathbf{q}) \succeq 0 $ is known as general inertia matrix, 
$C(\mathbf{q}, \dot{\mathbf{q}})$ is Coriolis-Centrifugal matrix, and 
$g(\mathbf{q})$ is potential forces impact (usually gravitational impact). 
For further convenience, I would omit the parameters of matrix functions. 
In the case of Coriolis-Centrifugal and potential forces, it is common to 
combine them in one term (bias force) $\mathbf{Q}(\mathbf{q}, \dot{\mathbf{q}}) = \boldsymbol{\tau}
- C(\mathbf{q}, \dot{\mathbf{q}}) \dot{\mathbf{q}} - g(\mathbf{q})$. Hence, I can 
rewrite \ref{eqn:cannonical_man_eq} in the following manner

\begin{equation}
    \label{eqn:compact_man_eq}
    M \ddot{\mathbf{q}} = \mathbf{Q}
\end{equation}

equation \ref{eqn:compact_man_eq} is compact form of equation 
\ref{eqn:cannonical_man_eq}. It would be highly utilized in further 
investigations.

Let's consider the same physical system again with imposed constraints at this time.
All holonomic constraints would be presented by 
$\varphi(\mathbf{q}, t): \mathbb{S}_{n_q} \times \mathbb{R} \rightarrow \mathbb{R}^m$, and 
all non-holonomic by 
$
\psi(\mathbf{q}, \dot{\mathbf{q}}, t): \mathbb{S}_{n_q} \times \mathbb{R}^{n_v} 
\times \mathbb{R} \rightarrow \mathbb{R}^k
$, where $\mathbb{S}_{n_q}$ generalized coordinates space in the most common case.
All constraints are satisfied if and only if the above functions are equal to zero.

These constraints can be rewritten in the form 
\begin{equation}
    \label{eqn:udwadia_const_form}
    A(\mathbf{q}, \dot{\mathbf{q}}, t) \ddot{\mathbf{q}} 
    - b(\mathbf{q}, \dot{\mathbf{q}}, t)
\end{equation}

by differentiation of holonomic constraints twice and non-holonomic once.
According to Udwadia-Kalaba the constrained force that would satisfy can be 
written

\begin{equation}
    \label{eqn:ud_ka_c_forces}
    Q_c = M^{1 / 2} (A M^{-1/2})^+(b - A M ^ {-1} Q)
\end{equation}

where $M^{\pm 1 / 2} = W \Lambda^{\pm 1/ 2} W^T$ ($W, \Lambda$ gain by 
eigendecomposition), $[*]^+$ is the Moore-Penrose inverse. Under these forces, 
the solution of the forward dynamics takes the form

\begin{equation}
    \label{eqn:forward_dyn_with_cf}
    \ddot{\mathbf{q}} = M^{-1}Q + M^{-1 / 2} (A M^{-1/2})^+(b - A M ^ {-1} Q)
\end{equation}

The equation \ref{eqn:ud_ka_c_forces} according to F. Udvadia and R. Kalaba
\cite{UdwadiaKalabaApproach} is the analytical solution to the minimization problem
based on Gauss's principle of least constraint

\begin{equation}
    \label{eqn:gauss_min_problem}
    \begin{aligned}
        \min_{\ddot{\mathbf{q}}} \quad &
        [\ddot{\mathbf{q}} - a]^T M [\ddot{\mathbf{q}} - a] \\
        \textrm{s.t.} \quad &
        A \ddot{\mathbf{q}} - b = 0 \\
        \quad &
        a(\mathbf{q}, \dot{\mathbf{q}}, t) = M^{-1} Q
    \end{aligned}
\end{equation}

While the solution proposed by F. Udvadia and R. Kalaba demonstrates precision, 
it is not without its drawbacks. Instability points and computationally intensive 
operations, such as the computation of matrix roots, present significant challenges. 
Moreover, a notable deficiency lies in the absence of prioritization within the 
methodology.

\section{Brief Lie theory}

The Lie group, a mathematical concept dating back to the 19th century, was first 
proposed by Sophus Lie, laying the foundation for continuous transformation groups. 
Although it was initially abstract, over time its influence has spread to various 
scientific and technological fields. Before proceeding further, it is necessary to 
consider the basics of Lie theory. The mathematical tools discussed in this section 
are crucial for the research discussed here.

Let $\mathcal{G}$ is smooth manifold that satisfies the group axioms. For any 
$\mathcal{X}$, $\mathcal{Y}$ and $\mathcal{Z}$ from $\mathcal{G}$ the following 
statements are true:

\begin{equation}
    \begin{aligned}
        \label{eqn:g_axi}
        & \mathcal{X} \circ \mathcal{Y} \in \mathcal{G} \quad && \text{(I)} \\
        & \exists \mathcal{E}:\ \mathcal{E} \circ \mathcal{X} 
        = \mathcal{X} \circ \mathcal{E} = \mathcal{X} \quad && \text{(II)} \\
        & \exists \mathcal{X}^{-1}:\ \mathcal{X}^{-1} \circ \mathcal{X}
        = \mathcal{X} \circ \mathcal{X}^{-1} = \mathcal{E} \quad && \text{(III)} \\ 
        & (\mathcal{X} \circ \mathcal{Y}) \circ \mathcal{Z} = 
        \mathcal{X} \circ (\mathcal{Y} \circ \mathcal{Z}) \quad && \text{(IV)}
    \end{aligned}
\end{equation}

For such group $T_{\mathcal{E}} \mathcal{G}$ is the Lia Algebra defined at 
$\mathcal{E}$ element. The geometric interpretation of algebra is tangent plane 
that touchs the smooth manifold (group). $\mathfrak{m} \equiv 
T_{\mathcal{E}} \mathcal{G}$ is always a vector space.

The next crucial defenition is a group action

\begin{equation}
    f: \mathcal{G} \times \mathcal{V} \to \mathcal{V}
\end{equation}

where $\mathcal{V}$ is some set. The operation defined above should satisfy 
the axioms ($v \in \mathcal{V}$; $\mathcal{X}, \mathcal{Y} \in \mathcal{G}$),

\begin{equation}
    \begin{aligned}
        \label{eqn:act_axi}
        & \mathcal{E} \cdot v = v \quad && \text{(I)} \\
        & (\mathcal{X} \circ \mathcal{Y}) \cdot v = 
        \mathcal{X} \cdot (\mathcal{Y} \cdot v) \quad && \text{(II)}
    \end{aligned}
\end{equation}

For instance, if the $\text{SO}(n)$ rotations is considered as Lie group, than the 
transformation $R \cdot x \equiv R x$ ($x \in \mathbb{R}^n$) is a group action.

The exponential map $\exp:\ \mathfrak{m} \to \mathcal{G}$ converts elements of 
Lie algebra to corresponding group. The log map do inverse operation. 
However, it is neccesary to remember that 
this map applicable only for "identity" algebra. However, there is a linear
transformation between $T_{\mathcal{X}} \mathcal{G}$ and $T_{\mathcal{E}} 
\mathcal{G}$. It is called adjoin.

It is known that $\mathfrak{m}$ is isomorphic to the vector space $\mathbb{R}^m$. 
One can write it as $\mathfrak{m} \cong \mathbb{R}^m$. Due to convinience this 
isomorphism would be highly utilized. Moreover, in the bellow sections only 
algebras constructed on $\mathbb{R}^m$ are considered. Thus, a maping between 
sets can be defined in the following manner (hat-vee notation)

\begin{equation}
    \begin{aligned}
        \label{eqn:hat_vee_not}
        & [*]^\wedge: \ \mathbb{R}^m \to \mathfrak{m} \quad &&
        \boldsymbol{\tau}^\wedge = \sum_{i=1}^{m} \tau_i E_i \\
        & [*]^\vee: \mathfrak{m} \to \mathbb{R}^m \quad && 
        \boldsymbol{\tau} = \sum_{i=1}^{m} \tau_i \mathbf{e}_i
    \end{aligned}
\end{equation}

where $\mathbf{e}_i$ are a base of $\mathbb{R}^m$ and $E_i$ are base vectors 
of $\mathfrak{m}$. Obviously, $\mathbf{e}_i^\wedge = E_i$. Using this mapping, the
exponential / log map can be modified 

\begin{equation}
    \begin{aligned}
        \label{eqn:exp_log_for_r_m}
        & \exp : \quad && \mathcal{X} = \exp(\boldsymbol{\tau}^\wedge) \\
        & \log : \quad && \boldsymbol{\tau} = \log(\mathcal{X})^\vee
    \end{aligned}
\end{equation}

Or, in the most convinient form

\begin{equation}
    \begin{aligned}
        \label{eqn:big_exp_log_for_r_m}
        & \text{Exp} : \quad && \mathcal{X} 
        = \exp(\boldsymbol{\tau}^\wedge) 
        = \text{Exp}(\boldsymbol{\tau})\\
        & \text{Log} : \quad && \boldsymbol{\tau} 
        = \log(\mathcal{X})^\vee
        = \text{Log}(\mathcal{X})
    \end{aligned}
\end{equation}

Through this definitions it easy to introduce plus and minus operations

\begin{equation}
    \begin{aligned}
        & \text{right-} \oplus: 
        \mathcal{X} \oplus \prescript{\mathcal{X}}{}{\boldsymbol{\tau}} 
        \equiv \mathcal{X} \circ 
        \text{Exp}(\prescript{\mathcal{X}}{}{\boldsymbol{\tau}}) \in \mathcal{G} \\
        & \text{right-} \ominus:
        \mathcal{Y} \ominus \mathcal{X} 
        \equiv \text{Log}(\mathcal{X}^{-1} \circ \mathcal{Y}) 
        \in T_{\mathcal{X}}\mathcal{G}
    \end{aligned}
\end{equation}

$\mathbf{TODO}$: continue section

\section{Defying constraints over multiple systems}

This section introduces a common holonomic constraint, which serves as a foundation 
for further investigations. This constraint delineates a rigid body's behavior, 
applicable to multiple manipulators.

Let $M_1, M_2, \dots, M_p$ are $p$ iinertia matrices for $p$ independent systems. 
The $i$-th system has $n_q^i$ generalized coordinates, $n_v^i$ generalized velocity 
components and $Q_i$ bias force. Thus, the common unconstrained dynamics is

\begin{equation}
    \label{eqn:common_dynamics}
    \begin{bmatrix}
        M_1 & 0   & \dots & 0 \\
        0   & M_2 & \dots & 0 \\
        \vdots & \vdots & \ddots & \vdots \\
        0   & 0   & \dots & M_p
    \end{bmatrix}
    \begin{bmatrix}
        \ddot{\mathbf{q}}_1 \\ \ddot{\mathbf{q}}_2 \\ \vdots \\ \ddot{\mathbf{q}}_p
    \end{bmatrix}
    = 
    \begin{bmatrix}
        Q_1 \\ Q_2 \\ \vdots \\ Q_p
    \end{bmatrix}
\end{equation}

The equation \ref{eqn:common_dynamics} can be rewriten in the manner of
\ref{eqn:compact_man_eq}. In such compact form it is convinient for futher analysis. 
Hense, let

\begin{equation}
    \label{eqn:common_mass}
    M_s = 
    \begin{bmatrix}
        M_1 & 0   & \dots & 0 \\
        0   & M_2 & \dots & 0 \\
        \vdots & \vdots & \ddots & \vdots \\
        0   & 0   & \dots & M_p
    \end{bmatrix}
\end{equation}

\begin{equation}
    \label{eqn:common_q}
    q_s = 
    \begin{bmatrix}
        \mathbf{q}_1 & \mathbf{q}_2 & \dots & \mathbf{q}_p
    \end{bmatrix}^T
\end{equation}

\begin{equation}
    \label{eqn:common_bias}
    Q_s = 
    \begin{bmatrix}
        Q_1 & Q_2 & \dots & Q_p
    \end{bmatrix}^T
\end{equation}

It implies to the following dynamics equation that describes motion of $p$ independent 
systems

\begin{equation}
    \label{eqn:common_eq}
    M_s \ddot{\mathbf{q}}_s = \mathbf{Q}_s
\end{equation}

Let $R_{ij}$ and $\mathbf{p}_{ij}$ are rotation and position of $j$-th frame attached to some 
link of the $i$-th system respectevly. The rigid body connection between two frames 
now can be defined as

\begin{numcases}{}
    R_{ij}R_d = R_{\alpha \beta} & \label{eqn:rot_const}
    \\
    (\mathbf{p}_{ij} - \mathbf{p}_{\alpha \beta})^T
    (\mathbf{p}_{ij} - \mathbf{p}_{\alpha \beta}) = l & \label{eqn:position_const}
\end{numcases}

where $R_d$ is a fixed rotation matrix and $l$ a distence between connection points 
inside a rigid body.

Substituting 
$\prescript{\alpha \beta}{ij}{\mathbf{d}} = \mathbf{p}_{ij} - \mathbf{p}_{\alpha \beta}$ 
and differencing respect to time twice the equation \ref{eqn:position_const}
transforms to 

\begin{equation}
    \label{eqn:pos_const_transform_1}
    \prescript{\alpha \beta}{ij}{\mathbf{d}}^T 
    \prescript{\alpha \beta}{ij}{\mathbf{d}} - l = 0
    \Rightarrow
    \prescript{\alpha \beta}{ij}{\mathbf{d}}^T 
    \prescript{\alpha \beta}{ij}{\dot{\mathbf{d}}} = 0
    \Rightarrow
    \prescript{\alpha \beta}{ij}{\mathbf{d}}^T 
    \prescript{\alpha \beta}{ij}{\ddot{\mathbf{d}}} + 
    \prescript{\alpha \beta}{ij}{\dot{\mathbf{d}}}^T 
    \prescript{\alpha \beta}{ij}{\dot{\mathbf{d}}} = 0
\end{equation}

The equation \ref{eqn:pos_const_transform_1} can be expresed via generalized coordinates

\begin{equation}
    \label{eqn:pos_const_transform_2}
    \prescript{\alpha \beta}{ij}{\mathbf{d}}^T 
    \prescript{\alpha \beta}{ij}{\ddot{\mathbf{d}}} + 
    \prescript{\alpha \beta}{ij}{\dot{\mathbf{d}}}^T 
    \prescript{\alpha \beta}{ij}{\dot{\mathbf{d}}} = 0
    \Rightarrow
    \prescript{\alpha \beta}{ij}{\mathbf{d}}^T 
    (
        \prescript{v}{ij}{J} \ddot{\mathbf{q}}_s + 
        \prescript{v}{ij}{\dot{J}} \dot{\mathbf{q}}_s
    ) + 
    \dot{\mathbf{q}}_s^T \prescript{v}{ij}{J}^T 
    \prescript{v}{ij}{J} \dot{\mathbf{q}}_s = 0
\end{equation}

where $\prescript{v}{ij}{J} \equiv \prescript{v}{ij}{J}(\mathbf{q}_s)$ is the 
$j$-th frame velocity Jacobian of the $i$-th system. Now, it is possible to 
convert \ref{eqn:position_const} constraint to \ref{eqn:udwadia_const_form} form

\begin{equation}
    \label{eqn:pos_const_a_matrix}
    A_p(\mathbf{q}_s, \dot{\mathbf{q}}_s) = 
    \prescript{\alpha \beta}{ij}{\mathbf{d}}^T \prescript{v}{ij}{J}
\end{equation}

\begin{equation}
    \label{eqn:pos_const_b_vector}
    b_p(\mathbf{q}_s, \dot{\mathbf{q}}_s) = 
    - \prescript{\alpha \beta}{ij}{\mathbf{d}}^T 
    \prescript{v}{ij}{\dot{J}} \dot{\mathbf{q}}_s
    - \dot{\mathbf{q}}_s^T \prescript{v}{ij}{J}^T
    \prescript{v}{ij}{J} \dot{\mathbf{q}}_s
\end{equation}

Now, let suppose that $\gamma_1, \gamma_2, \dots, \gamma_w$ system connected by rigid 
body. Here $\gamma_i$ is an index of the system, and $w \leq p$. 

\chapter{Methodology}
\label{chap:met}

This chapter will present a detailed elucidation of the principles underpinning the 
proposed methodology. The initial section will offer a concise overview of the 
Udvadia-Kalaba's approach is accompanied by a comprehensive exposition of the physical 
rationales employed in this technique. Subsequently, these principles will be 
harnessed in the development of the intended methodology. Furthermore, this section 
will scrutinize the limitations and distinctive characteristics inherent in the 
proposed methodology.

\section{Essentials of Udvadia-Kalaba approach}

Let $\mathbf{q}$ be a vector of generalized coordinates of some physical system.
Also let agree that $\dim \mathbf{q} = n_q$ and 
$\dim \dot{\mathbf{q}} = \dim \ddot{\mathbf{q}} = n_v$, where $\dot{\mathbf{q}}$ and 
$\ddot{\mathbf{q}}$ are generalized velocity and acceleration respectevly. Notably, 
that in the general case $n_q \neq n_v$ because $\dot{\mathbf{q}}$ and $\ddot{\mathbf{q}}$ 
can be located in different spaces. Thus, the dynamics of the given system can be 
expressed:

\begin{equation}
    \label{eqn:cannonical_man_eq}
    M(\mathbf{q}) \ddot{\mathbf{q}} 
    + C(\mathbf{q}, \dot{\mathbf{q}}) \dot{\mathbf{q}} 
    + g(\mathbf{q}) = \boldsymbol{\tau}
\end{equation}

where $M(\mathbf{q}) \succeq 0 $ is known as general inertia matrix, 
$C(\mathbf{q}, \dot{\mathbf{q}})$ is Coriolis-Centrifugal matrix, and 
$g(\mathbf{q})$ is potential forces impact (usually gravitational impact). 
For further convenience, I would omit the parameters of matrix functions. 
In the case of Coriolis-Centrifugal and potential forces, it is common to 
combine them in one term $\mathbf{Q}(\mathbf{q}, \dot{\mathbf{q}}) = \boldsymbol{\tau}
- C(\mathbf{q}, \dot{\mathbf{q}}) \dot{\mathbf{q}} - g(\mathbf{q})$. Hence, I can 
rewrite \ref{eqn:cannonical_man_eq} in the following manner

\begin{equation}
    \label{eqn:compact_man_eq}
    M \ddot{\mathbf{q}} = \mathbf{Q}
\end{equation}

equation \ref{eqn:compact_man_eq} is compact form of equation 
\ref{eqn:cannonical_man_eq}. It would be highly utilized in further 
investigations.

Let's consider the same physical system again with imposed constraints at this time.
All holonomic constraints would be presented by 
$\varphi(\mathbf{q}, t): \mathbb{S}_{n_q} \times \mathbb{R} \rightarrow \mathbb{R}^m$, and 
all non-holonomic by 
$
\psi(\mathbf{q}, \dot{\mathbf{q}}, t): \mathbb{S}_{n_q} \times \mathbb{R}^{n_v} 
\times \mathbb{R} \rightarrow \mathbb{R}^k
$, where $\mathbb{S}_{n_q}$ generalized coordinates space in the most common case.
All constraints are satisfied if and only if the above functions are equal to zero.

These constraints can be rewritten in the form 
$
A(\mathbf{q}, \dot{\mathbf{q}}, t) \ddot{\mathbf{q}} 
- b(\mathbf{q}, \dot{\mathbf{q}}, t)
$
by differentiation of holonomic constraints twice and non-holonomic once.
According to Udwadia-Kalaba the constrained force that would satisfy can be 
written

\begin{equation}
    \label{eqn:ud_ka_c_forces}
    Q_c = M^{1 / 2} (A M^{-1/2})^+(b - A M ^ {-1} Q)
\end{equation}

where $M^{\pm 1 / 2} = W \Lambda^{\pm 1/ 2} W^T$ ($W, \Lambda$ gain by 
eigendecomposition), $[*]^+$ is the Moore-Penrose inverse. Under these forces, 
the solution of the forward dynamics takes the form

\begin{equation}
    \label{eqn:forward_dyn_with_cf}
    \ddot{\mathbf{q}} = M^{-1}Q + M^{-1 / 2} (A M^{-1/2})^+(b - A M ^ {-1} Q)
\end{equation}

The equation \ref{eqn:ud_ka_c_forces} according to F. Udvadia and R. Kalaba
\cite{UdwadiaKalabaApproach} is the analytical solution to the minimization problem
based on Gauss's principle of least constraint

\begin{equation}
    \label{eqn:gauss_min_problem}
    \begin{aligned}
        \min_{\ddot{\mathbf{q}}} \quad &
        [\ddot{\mathbf{q}} - a]^T M [\ddot{\mathbf{q}} - a] \\
        \textrm{s.t.} \quad &
        A \ddot{\mathbf{q}} - b = 0 \\
        \quad &
        a(\mathbf{q}, \dot{\mathbf{q}}, t) = M^{-1} Q
    \end{aligned}
\end{equation}

While the solution proposed by F. Udvadia and R. Kalaba demonstrates precision, 
it is not without its drawbacks. Instability points and computationally intensive 
operations, such as the computation of matrix roots, present significant challenges. 
Moreover, a notable deficiency lies in the absence of prioritization within the 
methodology.

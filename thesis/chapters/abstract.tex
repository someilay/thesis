\begin{abstract}

\par The cooperation of robotic systems, particularly when constrained by shared 
objects, presents a significant challenge in various domains. Previous 
methods, such as models with pre-existing constraints or the 
Karush-Kuhn-Tucker (KKT) approach, suffer from computational complexity and 
prioritization issues. This research solves these limitations by proposing 
a novel methodology based on the Udwadia-Kalaba approach. 
The objective is to enhance control and collaboration of robots using 
advanced tools like MuJoCo, Pinocchio, and ProxSuite. The method integrates 
physical grounding and simplicity, allowing manual fine-tuning of 
constraint priorities and achieving improved computational speed. 
Experimental results demonstrate significant improvements in efficiency, 
accuracy, and adaptability in robotic operations. These advancements suggest 
potential for increased productivity and optimized performance in 
industrial and real-life applications, laying the way for further 
robotics technology developments.

\end{abstract}
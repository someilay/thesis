\chapter{Conclusion}
\label{chap:conclusion}

This research developed a straightforward methodology for 
coupling robotic systems constrained by a shared object using the 
Udwadia-Kalaba approach. This approach aimed to resolve the challenges of 
robustness, adaptiveness, and collaboration in various real-world applications of 
robotic systems. This study revealed that the proposed methodology achieves the 
stable control and interaction of robotic systems. These results are obtained 
by integrating modern physics simulation, robotic dynamics computation, and 
optimization libraries such as MuJoCo, Pinocchio, and ProxSuite. 
Specifically, the methodology demonstrated improved computational speed with 
keeping high precision in robotic operations. It allowed for manual fine-tuning of 
constraint priorities, facilitating flexibility and control over robotic 
collaboration. Through numerical experiments proved the methodology enhanced 
performance in coupling robots in collaborative work with solid objects. 
Especially, the described approach showed simplicity of creating and manipulating 
constraints.

This research provides a robust solution to the interacting complex 
robotic systems, which is critical for improving productivity and efficiency in 
various industrial and real-life applications. The integration of advanced 
computational tools enables more effective problem-solving, that consequently 
advances the capabilities of automated and robotic systems. The proposed 
methodology offers a novel combination of the Udwadia-Kalaba approach 
with modern simulation and optimization tools. It overcomes the limitations of 
previous methods such as the complexity and constraint prioritization issues. 
Futhermore, the suggested method is a practical and efficient solution 
for real-time interaction of multiple robots constrained by shared objects.

Despite its contributions, the research has several limitations. The study assumes 
systems of rigid bodies with at most inner stiffness, which may not fully 
represent all real-world robotic applications. Also, the approach requires a 
predefined description of the entire system, limiting its applicability in 
scenarios with changing or unknown system dynamics. Future research should 
focus on extending the methodology to handle non-rigid bodies and more complex 
physical interactions. Also, further studies should investigate finetuning $K_p$, 
$K_d$ parameters. These research should be aimed to two purposes, namely: 
enhancing the method's ability to work with wider range of physical systems, and 
conducting more extensive real-world experiments to validate 
the methodology's effectiveness. By addressing these areas, future research can 
further expand the applicability of this collaboration 
approach, laying the way for more advanced and efficient robotic systems.

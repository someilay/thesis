\appendix
\chapter{Brief Lie theory}

The Lie group, a mathematical concept dating back to the 19th century, was 
first proposed by Sophus Lie, laying the foundation for continuous 
transformation groups. Although it was initially abstract, over time its 
influence has spread to various scientific and technological fields. The 
mathematical tools discussed in this chapter are crucial for the research 
discussed here. It inspired by this study \cite{MicroLieTheory}, and 
contains only crucial information.

\section{Groups, Algebras, and operators}
\label{sec:group_algebra_ops}

Let $\mathcal{G}$ is smooth manifold that satisfies the group axioms. For any 
$\mathcal{X}$, $\mathcal{Y}$ and $\mathcal{Z}$ from $\mathcal{G}$ the following 
statements are true:

\begin{equation}
    \begin{aligned}
        \label{eqn:g_axi}
        & \mathcal{X} \circ \mathcal{Y} \in \mathcal{G} \quad && \text{(I)} \\
        & \exists \mathcal{E}:\ \mathcal{E} \circ \mathcal{X} 
        = \mathcal{X} \circ \mathcal{E} = \mathcal{X} \quad && \text{(II)} \\
        & \exists \mathcal{X}^{-1}:\ \mathcal{X}^{-1} \circ \mathcal{X}
        = \mathcal{X} \circ \mathcal{X}^{-1} = \mathcal{E} \quad && \text{(III)} \\ 
        & (\mathcal{X} \circ \mathcal{Y}) \circ \mathcal{Z} = 
        \mathcal{X} \circ (\mathcal{Y} \circ \mathcal{Z}) \quad && \text{(IV)}
    \end{aligned}
\end{equation}

For such group $T_{\mathcal{E}} \mathcal{G}$ is the Lia Algebra defined at 
$\mathcal{E}$ element. The geometric interpretation of algebra is tangent plane 
that touches the smooth manifold (group). $\mathfrak{m} \equiv 
T_{\mathcal{E}} \mathcal{G}$ is always a vector space.

The next crucial definition is a group action

\begin{equation}
    f: \mathcal{G} \times \mathcal{V} \to \mathcal{V}
\end{equation}

where $\mathcal{V}$ is some set. The operation defined above should satisfy 
the axioms ($v \in \mathcal{V}$; $\mathcal{X}, \mathcal{Y} \in \mathcal{G}$),

\begin{equation}
    \begin{aligned}
        \label{eqn:act_axi}
        & \mathcal{E} \cdot v = v \quad && \text{(I)} \\
        & (\mathcal{X} \circ \mathcal{Y}) \cdot v = 
        \mathcal{X} \cdot (\mathcal{Y} \cdot v) \quad && \text{(II)}
    \end{aligned}
\end{equation}

For instance, if the $\text{SO}(n)$ rotations is considered as Lie group, than the 
transformation $R \cdot x \equiv R x$ ($x \in \mathbb{R}^n$) is a group action.

The exponential map $\exp:\ \mathfrak{m} \to \mathcal{G}$ converts elements of 
Lie algebra to corresponding group. The log map do inverse operation. 
However, it is necessary to remember that 
this map applicable only for "identity" algebra. However, there is a linear
transformation between $T_{\mathcal{X}} \mathcal{G}$ and $T_{\mathcal{E}} 
\mathcal{G}$. It is called adjoin.

It is known that $\mathfrak{m}$ is isomorphic to the vector space $\mathbb{R}^m$. 
One can write it as $\mathfrak{m} \cong \mathbb{R}^m$. Due to convenience this 
isomorphism would be highly utilized. Moreover, in the bellow sections only 
algebras constructed on $\mathbb{R}^m$ are considered. Thus, a mapping between 
sets can be defined in the following manner (hat-vee notation)

\begin{equation}
    \begin{aligned}
        \label{eqn:hat_vee_not}
        & [*]^\wedge: \ \mathbb{R}^m \to \mathfrak{m} \quad &&
        \boldsymbol{\tau}^\wedge = \sum_{i=1}^{m} \tau_i E_i \\
        & [*]^\vee: \mathfrak{m} \to \mathbb{R}^m \quad && 
        \boldsymbol{\tau} = \sum_{i=1}^{m} \tau_i \mathbf{e}_i
    \end{aligned}
\end{equation}

where $\mathbf{e}_i$ are a base of $\mathbb{R}^m$ and $E_i$ are base vectors 
of $\mathfrak{m}$. Obviously, $\mathbf{e}_i^\wedge = E_i$. Using this mapping, the
exponential / log map can be modified 

\begin{equation}
    \begin{aligned}
        \label{eqn:exp_log_for_r_m}
        & \exp : \quad && \mathcal{X} = \exp(\boldsymbol{\tau}^\wedge) \\
        & \log : \quad && \boldsymbol{\tau} = \log(\mathcal{X})^\vee
    \end{aligned}
\end{equation}

Or, in the most convenient form

\begin{equation}
    \begin{aligned}
        \label{eqn:big_exp_log_for_r_m}
        & \text{Exp} : \quad && \mathcal{X} 
        = \exp(\boldsymbol{\tau}^\wedge) 
        = \text{Exp}(\boldsymbol{\tau})\\
        & \text{Log} : \quad && \boldsymbol{\tau} 
        = \log(\mathcal{X})^\vee
        = \text{Log}(\mathcal{X})
    \end{aligned}
\end{equation}

Through this definitions it easy to introduce plus and minus operations

\begin{equation}
    \begin{aligned}
        & \text{right-} \oplus: &&
        \mathcal{X} \oplus \prescript{\mathcal{X}}{}{\boldsymbol{\tau}} 
        \equiv \mathcal{X} \circ 
        \text{Exp}(\prescript{\mathcal{X}}{}{\boldsymbol{\tau}}) \in 
        \mathcal{G} \\
        & \text{right-} \ominus: &&
        \mathcal{Y} \ominus \mathcal{X} 
        \equiv \text{Log}(\mathcal{X}^{-1} \circ \mathcal{Y}) 
        \in T_{\mathcal{X}}\mathcal{G} \\
        & \text{left-} \oplus: &&
        \prescript{\mathcal{E}}{}{\boldsymbol{\tau}} \oplus \mathcal{X}
        \equiv \text{Exp}(\prescript{\mathcal{E}}{}{\boldsymbol{\tau}}) 
        \circ \mathcal{X} \in \mathcal{G} \\ 
        & \text{left-} \ominus: &&
        \mathcal{Y} \ominus \mathcal{X} \equiv
        \text{Log}(\mathcal{Y} \circ \mathcal{X}^{-1}) 
        \in T_{\mathcal{E}}\mathcal{G}
    \end{aligned}
\end{equation}

Using the above statements it is possible to define Jacobians

\begin{equation}
    \begin{aligned}
        & J_r(\boldsymbol{\tau}): &&
        \text{Exp}(\boldsymbol{\tau} + \delta \boldsymbol{\tau})
        \approx
        \text{Exp}(\boldsymbol{\tau}) \oplus 
        [J_r(\boldsymbol{\tau}) \delta \boldsymbol{\tau}] \\ 
        & J_r^{-1}(\boldsymbol{\tau}): &&
        \text{Log}(\mathcal{X} \oplus \delta \boldsymbol{\tau}) 
        \approx
        \text{Log}(\mathcal{X}) + 
        J_r^{-1}(\boldsymbol{\tau}) \delta \boldsymbol{\tau}
    \end{aligned}
\end{equation}

where $\mathcal{X} = \text{Exp}(\boldsymbol{\tau})$, and $\oplus$, $\ominus$ 
are right versions. The above definitions describe right version of 
Jacobians. There are also left ones that based on left minus and plus 
operators.

\section{Special orthogonal group}

Let's consider $\SO$ (Special orthogonal group). It consists from orthogonal 
rotations matrices, and forms a group under multiplication. According to 
Euler's rotation theorem any orientation can be described by rotation 
about some single axis with precision to $2 \pi n$. Thus, $\SO$ can be 
considered as Lie group, and $\mathbb{R}^3$ is a Lie algebra respectively.

Now let's define a wedge operator $[.]^{\wedge}$:

\begin{equation}
    \hat{\mathbf{a}} = \mathbf{a}^{\wedge} = 
    \begin{bmatrix}
        0 & -a_3 & a_2 \\
        a_3 & 0 & -a_1 \\
        -a_2 & a_1 & 0 \\
    \end{bmatrix}
    \label{eqn:wedge_op}
\end{equation}

where $\mathbf{a} \in \mathbb{R}^3$, and vee operator ($[.]^{\vee}$) is 
following

\begin{equation}
    A^{\vee} = \mathbf{a}, \ \text{where }
    \hat{\mathbf{a}} = A
    \label{eqn:vee_op}
\end{equation}

Using these definitions it is possible to connect $\dot{R}$ with classical 
angular velocity

\begin{equation}
    \begin{aligned}
        \hat{\boldsymbol{\omega}}^s = \dot{R} R^T & \quad
        \hat{\boldsymbol{\omega}}^b = R^T \dot{R}
    \end{aligned}
    \label{eqn:ang_with_dot_r}
\end{equation}

where $b$ and $s$ subscripts denote body and spatial velocities 
respectively.

The exponential map is

\begin{equation}
    R = \text{Exp}(\boldsymbol{\theta}) \equiv
    I + \frac{\sin \theta}{\theta} \hat{\boldsymbol{\theta}} +
    \frac{1 - \cos \theta}{\theta^2} \hat{\boldsymbol{\theta}}^2
    \label{eqn:exp_map}
\end{equation}

where $\boldsymbol{\theta} = \theta \mathbf{u}$, and $\mathbf{u}$ is 
arbitrary unit vector from $\mathbb{R}^3$. Therefore, the logarithm map is

\begin{equation}
    \boldsymbol{\theta} = \text{Log}(R) \equiv 
    \frac{\theta (R - R^T)^{\vee}}{2 \sin \theta}, \ 
    \text{where } 
    \theta = \cos^{-1} \frac{\text{trace}(R) - 1}{2}
    \label{eqn:log_map}
\end{equation}

In the above statement $R - R^T$ is skew symmetric, it can be proved by 
transposing it. 

Now let's move to left and right Jacobians definitions 

\begin{equation}
    \begin{aligned}
        J_r(\boldsymbol{\theta}) & = 
        I - \frac{1 - \cos \theta}{\theta^2} \hat{\boldsymbol{\theta}} +
        \frac{\theta - \sin \theta}{\theta^3} \hat{\boldsymbol{\theta}}^2 \\
        J_r^{-1}(\boldsymbol{\theta}) & = 
        I + \frac{1}{2} \hat{\boldsymbol{\theta}} + 
        \biggl( \frac{1}{\theta^2} - 
        \frac{1 + \cos \theta}{2 \theta \sin \theta} \biggr)
        \hat{\boldsymbol{\theta}}^2 \\ 
        J_l(\boldsymbol{\theta}) & = J_r^T(\boldsymbol{\theta}) \\
        J_l^{-1}(\boldsymbol{\theta}) & = J_r^{-T}(\boldsymbol{\theta})
    \end{aligned}
    \label{eqn:l_and_r_jacobians}
\end{equation}

The all of the above definitions are reformulated from Appendix B in 
\cite{MicroLieTheory}. The mathematical ground for them can be found 
in \cite{LieStochModels}.

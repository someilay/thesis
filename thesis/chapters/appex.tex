\appendix
\chapter{Brief Lie theory}

The Lie group, a mathematical concept dating back to the 19th century, was first 
proposed by Sophus Lie, laying the foundation for continuous transformation groups. 
Although it was initially abstract, over time its influence has spread to various 
scientific and technological fields. Before proceeding further, it is necessary to 
consider the basics of Lie theory. The mathematical tools discussed in this section 
are crucial for the research discussed here.

Let $\mathcal{G}$ is smooth manifold that satisfies the group axioms. For any 
$\mathcal{X}$, $\mathcal{Y}$ and $\mathcal{Z}$ from $\mathcal{G}$ the following 
statements are true:

\begin{equation}
    \begin{aligned}
        \label{eqn:g_axi}
        & \mathcal{X} \circ \mathcal{Y} \in \mathcal{G} \quad && \text{(I)} \\
        & \exists \mathcal{E}:\ \mathcal{E} \circ \mathcal{X} 
        = \mathcal{X} \circ \mathcal{E} = \mathcal{X} \quad && \text{(II)} \\
        & \exists \mathcal{X}^{-1}:\ \mathcal{X}^{-1} \circ \mathcal{X}
        = \mathcal{X} \circ \mathcal{X}^{-1} = \mathcal{E} \quad && \text{(III)} \\ 
        & (\mathcal{X} \circ \mathcal{Y}) \circ \mathcal{Z} = 
        \mathcal{X} \circ (\mathcal{Y} \circ \mathcal{Z}) \quad && \text{(IV)}
    \end{aligned}
\end{equation}

For such group $T_{\mathcal{E}} \mathcal{G}$ is the Lia Algebra defined at 
$\mathcal{E}$ element. The geometric interpretation of algebra is tangent plane 
that touchs the smooth manifold (group). $\mathfrak{m} \equiv 
T_{\mathcal{E}} \mathcal{G}$ is always a vector space.

The next crucial defenition is a group action

\begin{equation}
    f: \mathcal{G} \times \mathcal{V} \to \mathcal{V}
\end{equation}

where $\mathcal{V}$ is some set. The operation defined above should satisfy 
the axioms ($v \in \mathcal{V}$; $\mathcal{X}, \mathcal{Y} \in \mathcal{G}$),

\begin{equation}
    \begin{aligned}
        \label{eqn:act_axi}
        & \mathcal{E} \cdot v = v \quad && \text{(I)} \\
        & (\mathcal{X} \circ \mathcal{Y}) \cdot v = 
        \mathcal{X} \cdot (\mathcal{Y} \cdot v) \quad && \text{(II)}
    \end{aligned}
\end{equation}

For instance, if the $\text{SO}(n)$ rotations is considered as Lie group, than the 
transformation $R \cdot x \equiv R x$ ($x \in \mathbb{R}^n$) is a group action.

The exponential map $\exp:\ \mathfrak{m} \to \mathcal{G}$ converts elements of 
Lie algebra to corresponding group. The log map do inverse operation. 
However, it is neccesary to remember that 
this map applicable only for "identity" algebra. However, there is a linear
transformation between $T_{\mathcal{X}} \mathcal{G}$ and $T_{\mathcal{E}} 
\mathcal{G}$. It is called adjoin.

It is known that $\mathfrak{m}$ is isomorphic to the vector space $\mathbb{R}^m$. 
One can write it as $\mathfrak{m} \cong \mathbb{R}^m$. Due to convinience this 
isomorphism would be highly utilized. Moreover, in the bellow sections only 
algebras constructed on $\mathbb{R}^m$ are considered. Thus, a maping between 
sets can be defined in the following manner (hat-vee notation)

\begin{equation}
    \begin{aligned}
        \label{eqn:hat_vee_not}
        & [*]^\wedge: \ \mathbb{R}^m \to \mathfrak{m} \quad &&
        \boldsymbol{\tau}^\wedge = \sum_{i=1}^{m} \tau_i E_i \\
        & [*]^\vee: \mathfrak{m} \to \mathbb{R}^m \quad && 
        \boldsymbol{\tau} = \sum_{i=1}^{m} \tau_i \mathbf{e}_i
    \end{aligned}
\end{equation}

where $\mathbf{e}_i$ are a base of $\mathbb{R}^m$ and $E_i$ are base vectors 
of $\mathfrak{m}$. Obviously, $\mathbf{e}_i^\wedge = E_i$. Using this mapping, the
exponential / log map can be modified 

\begin{equation}
    \begin{aligned}
        \label{eqn:exp_log_for_r_m}
        & \exp : \quad && \mathcal{X} = \exp(\boldsymbol{\tau}^\wedge) \\
        & \log : \quad && \boldsymbol{\tau} = \log(\mathcal{X})^\vee
    \end{aligned}
\end{equation}

Or, in the most convinient form

\begin{equation}
    \begin{aligned}
        \label{eqn:big_exp_log_for_r_m}
        & \text{Exp} : \quad && \mathcal{X} 
        = \exp(\boldsymbol{\tau}^\wedge) 
        = \text{Exp}(\boldsymbol{\tau})\\
        & \text{Log} : \quad && \boldsymbol{\tau} 
        = \log(\mathcal{X})^\vee
        = \text{Log}(\mathcal{X})
    \end{aligned}
\end{equation}

Through this definitions it easy to introduce plus and minus operations

\begin{equation}
    \begin{aligned}
        & \text{right-} \oplus: 
        \mathcal{X} \oplus \prescript{\mathcal{X}}{}{\boldsymbol{\tau}} 
        \equiv \mathcal{X} \circ 
        \text{Exp}(\prescript{\mathcal{X}}{}{\boldsymbol{\tau}}) \in \mathcal{G} \\
        & \text{right-} \ominus:
        \mathcal{Y} \ominus \mathcal{X} 
        \equiv \text{Log}(\mathcal{X}^{-1} \circ \mathcal{Y}) 
        \in T_{\mathcal{X}}\mathcal{G}
    \end{aligned}
\end{equation}

$\mathbf{TODO}$: continue section

\chapter{Literature Review}
\label{chap:lr}
% \chaptermark{Second Chapter Heading}

The control of the interacting physical systems is a challenging 
task. There are few problems that should be solved to achieve the 
high efficiency and precision in the aforementioned problem. They 
are the following: the right mathematical model selection, the 
unified methodology for defining the interaction, and the well-defined 
control error. The first point is necessary to cover a wide range 
of physical systems. The second one is crucial to work with 
different mechanical connections. Finally, the last one is needed 
to stability and robustness analysis. 

This literature review covers all these subproblems and explores them
from the point of view of the work of Firdaus Udwadia and Robert Kalaba 
\cite{UdwadiaKalabaApproach}. The section \ref{sec:math_model} considers 
a mathematical models that utilized in the recent studies. The 
\ref{sec:interaction_def} part reviews a rigorous methodology to define 
how physical systems can affect each other. The section \ref{sec:control_error} 
explores the previous research in control error defining and analysis 
on different manifolds.

The articles were selected to demonstrate a solid ground why some methods 
had been preferred in this study. Some of the papers were chosen 
to reference for theoretical background that was not included in this 
paper. Other articles had been reviewed for shown an existing techniques.

\section{The mathematical model} \label{sec:math_model}

This section contains review of existing mathematical models for physical 
systems. Moreover, this piece has a comparison of them in terms of numerical 
integration convenience and simplicity of defining the initial conditions. 
It is necessary to clarify that from now this study considers only systems 
of rigid body with only inner stiffness. The detailed explanation of 
such choice will be conduct later in the Chapter \ref{chap:met}.

Firstly, lets consider the most popular model. The articles 
\cite{UdwadiaKalabaApproach}, \cite{AppOfUdwadiaKalabaApproach} and 
\cite{UnifiedFrameworkOfRobotControl} relies on it. Usually it is 
called the canonical manipulator equation. The model can be formulated 
as 

\begin{equation} \label{eqn:can_man_equation}
    M(\mathbf{q})\ddot{\mathbf{q}} + 
    C(\mathbf{q}, \dot{\mathbf{q}}) \dot{\mathbf{q}} + 
    g(\mathbf{q}) = 
    \boldsymbol{\tau}
\end{equation}

where $M(\mathbf{q})$ is known as inertia matrix, $C(\mathbf{q}, \dot{\mathbf{q}})$ 
is Centrifugal-Coriolis matrix, $g(\mathbf{q})$ is a gradient
of potential forces, and $\mathbf{q}$, $\boldsymbol{\tau}$ are generalized 
coordinates and torques respectively. This equation will be explored in details 
in the next Chapter \ref{chap:met}.

Conversely, Udwadia \cite{ConstHamiltonSys} utilized a Hamiltonian view of the 
problem. The equations are following, 

\begin{equation}
    \label{eqn:ham_eqns}
    \begin{cases}
        \dot{\mathbf{q}} = \frac{\partial \mathcal{H}}{\partial \mathbf{p}} \\
        \dot{\mathbf{p}} = -\frac{\partial \mathcal{H}}{\partial \mathbf{q}}
    \end{cases}
\end{equation}

where $\mathbf{q}$, $\mathbf{p}$, and $\mathcal{H}$ are generalized coordinates,
generalized momentum, and Hamiltonian respectively.

The aforementioned equations are proven to be equivalent. Thus, both models 
can be utilized for achieving the main goal. However, the crucial difference 
lies in the another plane. 

In the vast majority of cases the both differential equations can be solved 
analytically. Therefore, it is necessary to use the numerical methods. The 
equation (\ref{eqn:can_man_equation}) has second time derivative. It forces 
to construct proper state variable for numerical integration. In the opposite 
the Hamiltonian equations (\ref{eqn:ham_eqns}) has only first derivative. Hence, 
the usage of it is simpler. However, the definition of initial conditions is harder.
Nevertheless, the mentioned problem are already solved. So, choice of model 
fully lies on the specific of discussed task. 

The most of reviewed papers relies on the canonical model 
(\ref{eqn:can_man_equation}). Therefore, this research will utilize this equation 
too.

\section{The methodology to defining interaction} \label{sec:interaction_def}

The second crucial point in this paper is a definition of interaction between 
systems. This piece considers how it can be achieved, and which methodology 
is better in the context of discussed question. The important remark of this 
section that only interaction via rigid bodies will be consider below.

The first discussed approach is demonstrated by Scott Kuindersma and et al. in
\cite{OptimizationBasedLocomotionPlanning}. This research propose to use a 
a force cone to define an contact between a physical system (Atlas in their case) 
and solid surface. The actuation force is formulated as 

\begin{equation}
    \label{eqn:force_cone}
    \pmb{\lambda} = \sum_{i=1}^{N_d} \beta_i (\mathbf{n} + \mu \mathbf{d}_i), 
    \: \beta_i \ge 0
\end{equation}

where $\mathbf{n}$ is a normal force, $\mathbf{d}_i$ is a tangent to contact 
vector, $\mu$ is the Coulomb friction coefficient, and $N_d$ is a amount
of used tangent vectors. This contact force later can be translated to 
joint space via respective jacobian - $J^T(\mathbf{q})$.

\textcolor{red}{TO BE CONTINUED}

\section{Control error} \label{sec:control_error}

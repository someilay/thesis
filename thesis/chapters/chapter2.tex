\chapter{Literature Review}
\label{chap:lr}
% \chaptermark{Second Chapter Heading}


The control of the interacting physical systems is a challenging
task. To achieve high efficiency and precision in the said task, three problems 
should be solved, namely: the choice of the most appropriate mathematical model, the
unified methodology for defining the interaction, and the well-defined
virtual constraint. The first element is necessary to cover a wide range
of physical systems. The second one is crucial to work with
different mechanical connections. The last one is needed
for stability and robustness analysis.


This literature review covers all these subproblems and explores them
from the point of view of the work of Firdaus Udwadia and Robert Kalaba
\cite{UdwadiaKalabaApproach}. Section \ref{sec:math_model} considers
a mathematical model was used in the recent studies. Section
\ref{sec:interaction_def} reviews a rigorous methodology to define
how physical systems can affect each other. Section \ref{sec:virtual_constraint}
explores previous research in virtual constraint defining and analysis
of different manifolds. Last Section \ref{sec:lr_conclusion}
concludes this chapter and outlines the techniques that were chosen for 
further investigation.


The selection of the papers for this review aims to provide solid grounds for
the choice of the methods in this study. Some of the papers were chosen as a
reference for theoretical background. Other papers were reviewed to study the
existing techniques.


\section{The mathematical model} \label{sec:math_model}


This section contains a review of existing mathematical models for physical
systems. It also compares the models in terms of numerical
integration convenience and simplicity of defining the initial conditions.
It is necessary to clarify that from now on this study considers only systems
of rigid bodies with only inner stiffness. The detailed explanation of the reasons 
behind such a choice is presented later in Chapter \ref{chap:met}.


The most popular relevant model is usually referred as canonical manipulator 
equation \cite{UdwadiaKalabaApproach}, \cite{AppOfUdwadiaKalabaApproach}, and
\cite{UnifiedFrameworkOfRobotControl}. Usually, it is
called the canonical manipulator equation. The model can be formulated
as


\begin{equation} \label{eqn:can_man_equation}
   M(\mathbf{q})\ddot{\mathbf{q}} +
   C(\mathbf{q}, \dot{\mathbf{q}}) \dot{\mathbf{q}} +
   g(\mathbf{q}) =
   \boldsymbol{\tau}
\end{equation}


where $M(\mathbf{q})$ is known as inertia matrix, $C(\mathbf{q}, \dot{\mathbf{q}})$
is Centrifugal-Coriolis matrix, $g(\mathbf{q})$ is a gradient
of potential forces, and $\mathbf{q}$, $\boldsymbol{\tau}$ are generalized
coordinates and torques respectively. This equation is explored in detail
in next Chapter \ref{chap:met}.


Conversely, Udwadia \cite{ConstHamiltonSys} utilized a Hamiltonian view of the
problem. This equation also manipulates deals with generalized coordinates but
additionally it introduces generalized momentum and Hamiltonian. The choice of
such variables makes this approach more convenient in terms of numerical
integration.


The aforementioned equations are proven to be equivalent. Thus, both models
can be utilized to achieve the main goal. However, the crucial difference
lies in the other plane.


In the vast majority of cases, both differential equations can be solved
analytically. Therefore, it is necessary to use the numerical methods. Equation 
(\ref{eqn:can_man_equation}) has a second-time derivative. It enforces the 
construction of proper state variables for numerical integration. Conversely, 
the Hamiltonian equations have only the first derivative. Hence, they are more 
simple to use. Although the definition of initial conditions is harder for 
Hamiltonian equations, this problem has already been solved. So, the choice of 
the model fully depends on the specific discussed task.


Most of the reviewed papers rely on the canonical model
(\ref{eqn:can_man_equation}). Therefore, for our project I adhere to 
the same equation.


\section{The methodology to defining coupling} \label{sec:interaction_def}


The second crucial problem in this paper is the definition of coupling between
systems. In this section I show how this problem can be addressed and 
which methodology is preferable in the context of the discussed question. Importantly, 
here I only consider interaction through rigid bodies. The schematic representation 
of such coupling is shown in Fig. \ref{fig:rigid_coupling}. Chapter \ref{chap:met} 
presents a reasons why such coupling is chosen.


\begin{figure}[H]
   \centering
   \begin{tikzpicture}[
       media/.style={font={\footnotesize\sffamily}},
       interface/.style={
           % The border decoration is a path replacing decorator.
           % For the interface style we want to draw the original path.
           % The postaction option is therefore used to ensure that the
           % border decoration is drawn *after* the original path.
           postaction={draw,decorate,decoration={border,angle=-45,
                       amplitude=0.3cm,segment length=2mm}}},
       ]
       % Draw first manipulator
       \draw[black, ultra thick](0, 0) -- (0.5, 1.8);
       \draw[black, ultra thick](0.5, 1.8) -- (2.5, 2.2);
       % Gripper
       \gripper{2.5}{2.2}{3.3}{2.36}
       % Draw second point mass
       \drawpointmasssimple{0.5}{1.8}
       \base{0}{0}


       % Draw body
       \rectangle{2.737}{2.035}{2.655}{2.446}{4.851}{2.882}


       % Draw the second manipulator
       \draw[black, ultra thick](8.5, 0.8) -- (7.088, 3.117);
       \draw[black, ultra thick](7.088, 3.117) -- (5.088, 2.717);
       % Gripper
       \gripper{5.088}{2.717}{4.288}{2.557}
       % Draw second point mass
       \drawpointmasssimple{7.088}{3.117}
       \base{8.5}{0.8}
   \end{tikzpicture}
   \caption{Rigid body coupling}
   \label{fig:rigid_coupling}
\end{figure}


The first straightforward solution of this problem is deliberately include 
interaction in the mathematical model. In the discussed case it can be achieved 
by deriving $M(\mathbf{q})$, $C(\mathbf{q}, \dot{\mathbf{q}})$, and 
$g(\mathbf{q})$ in equation (\ref{eqn:can_man_equation}) as formulation of 
closed-loop system. It can be achieved if it is possible to formulate and 
analytically solve the following equation:


\begin{equation}
   \label{eqn:holonom_const}
   \varphi (\mathbf{q}, t) = 0
\end{equation}


From \ref{eqn:holonom_const}, the minimal set of generalized coordinates can be 
constructed, which implies rewriting the aforementioned parts of the canonical 
manipulator equation. Thus, the advantage of this method is the absence of the need 
for stabilization, because it is guaranteed by the model itself. Furthermore, this 
formulation can be used with a great range of control techniques. However, 
\cite{BodyDynWithClosedLoop} demonstrates a lower computation speed of terms in 
comparison to open-loop algorithms \cite{Pinocchio}. Moreover, proposed closed-loop
version requires a predefined description of whole system and works well
only with systems that have a small number of DoFs. Thus, it cannot
be used in real time in a dynamic environment.


The second discussed approach is analyzed in \cite{OptimizationBasedLocomotionPlanning}
and \cite{WholeBodyControlForm}. These articles propose to use a
force cone to define a contact between a physical system
and a solid surface. The actuation force can be formulated as


\begin{equation}
   \label{eqn:force_cone}
   \pmb{\lambda} = \sum_{i=1}^{N_d} \beta_i (\mathbf{n} + \mu \mathbf{d}_i),
   \: \beta_i \ge 0
\end{equation}


where $\mathbf{n}$ is a normal force, $\mathbf{d}_i$ is a tangent to contact
vector, $\mu$ is the Coulomb friction coefficient, and $N_d$ is the amount
of used tangent vectors. This contact force later can be translated to
joint space via respective Jacobian - $J^T(\mathbf{q})$.

This approach allows to emulate an interaction between
physical systems via a rigid body. It can be achieved by defining the motion
of the connecting body through the force acting on it. However, this method cannot
guarantee stability. Moreover, constructing a feedback loop in this case is
a complex task.

Finally, the third approach to solve the discussed problem is defining the right 
constraint. In this study the impact of the rigid body on connected systems can 
be described by holonomic constraint, which is described by equation 
(\ref{eqn:holonom_const}). Further, this equation can be used in the KKT 
(Karush-Kuhn-Tucker) technique to define a system with imposed constraints. 
Moreover, the equation (\ref{eqn:holonom_const}) facilitates the construction of a 
stabilization mechanism. For instance, Baumgarte's approach can be used 
with the KKT method to achieve this goal. Nevertheless, this study does not rely on 
the aforementioned technique (KKT), instead, I utilized the Udwadia-Kalaba approach, 
which rewrites generalized accelerations obtained from equation
(\ref{eqn:holonom_const}) as affine transform. Details are shown in
Chapter \ref{chap:met}.

\section{Virtual constraint} \label{sec:virtual_constraint}

The last crucial subproblem is the rigorous definition of virtual constraints to 
ensure stabilization. It is especially important in the context of the chosen type of
interaction. Therefore, this section reviews the ways to calculate the discrepancy
between the current system state and the desired one in the form of equation
(\ref{eqn:holonom_const}) and explains the ways to to guarantee its convergence in the
context of the form proposed by the Udwadia-Kalaba approach.

In the initial study \cite{UdwadiaKalabaApproach} Ferdaus Udwadia and Robert
Kalaba state that the straightforward differentiation of the holonomic
constraint (\ref{eqn:holonom_const}) produces instability during numerical
integration (constraint drift). Thus the authors recommend to use
Baumgarte's stabilization \cite{BaumgarteStab} via rewriting the linear second
order differential equation with $\varphi$ in the proposed matrix form.

Nevertheless, further in this study I show that it is hard to use
this technique in the case of attitude tracking. The problem mostly
is caused by the fact that the linear differential equations cannot
capture the topology of the $\text{SE}(3)$. Therefore, I was forced
to use non-linear ones by applying the right pose difference formulation.
Studies \cite{SlidingOnManifoldsQuat,GeomControlQuadSE3,
RigidBodyAttCon,OutFeedbackStabForOrbRob,ANonlinearObserverUsingPose}
describe methods to achieve this. The majority \cite{GeomControlQuadSE3,
RigidBodyAttCon,OutFeedbackStabForOrbRob,ANonlinearObserverUsingPose} of
source explores the problem through utilizing rotations matrices. On the other hand, 
\cite{SlidingOnManifoldsQuat} uses quaternions for achieving the goal.
Let's compare these solutions.

The study \cite{SlidingOnManifoldsQuat} states the advantage of quaternions over 
the rotations matrix. This research shows that the proposed method achieves
the \emph{exponential asymptotic stability} against \emph{almost global
stability} of techniques from \cite{GeomControlQuadSE3,RigidBodyAttCon}.
Furthermore, the quaternion-based method in the discussed study 
\cite{SlidingOnManifoldsQuat} presents the superiority of the analogous ones. The 
authors prove that their technique avoids the unwinding phenomenon, i.e., error 
converges to zero by the shortest path in $\mathbb{S}^3$.

The solution via utilizing the rotations matrices is presented in
\cite{GeomControlQuadSE3,RigidBodyAttCon,OutFeedbackStabForOrbRob,
ANonlinearObserverUsingPose}. However, the aforementioned quaternion-based
approach shows that basing on such structure methods has disadvantages.
The naive error computation via matrices is a calculation of dot product
between base vectors of current and desired orientation. As mentioned
above, this approach can achieve only \emph{almost global stability}.
Nevertheless, studies \cite{OutFeedbackStabForOrbRob, ANonlinearObserverUsingPose}
rely on deep topology analysis of $SE(3)$ group. These research use 
Lie theory to construct a right discrepancy between attitudes. In
Chapter \ref{chap:met} the \emph{exponential convergence} of the method
based on the studies presented in \cite{OutFeedbackStabForOrbRob} and 
\cite{ANonlinearObserverUsingPose} is proven.

To conclude, in this study the rotation matrices are chosen to define an
attitude error. My method draws upon the approach presented in
\cite{OutFeedbackStabForOrbRob} and \cite{ANonlinearObserverUsingPose}.
The main reason for such a choice is the convenience of work in the context of
framework \cite{Pinocchio}, which I use for numerical experiments.

\section{Conclusion} \label{sec:lr_conclusion}

In this chapter, the solutions to the necessary aforementioned subproblems
were reviewed. The canonical manipulator equation (\ref{eqn:can_man_equation})
was chosen as the base for investigations because it is most widely used in previous 
studies. To define coupling between systems, I decided to
use the Udwadia-Kalaba approach and the form of coupling proposed in the latter.
Finally, for building a stable virtual constraint the discrepancy based
on Lie theory was taken.




\chapter{Introduction}
\label{chap:intro}
% \chaptermark{Optional running chapter heading}

The advancement of microelectronics and sensor technologies has catalyzed a 
widespread integration of robotic systems into various domains. Manipulators, 
delivery robots, and flying drones have become ubiquitous, presenting developers 
and researchers with novel challenges in terms of robustness, adaptiveness, and 
synchronization. Among these challenges, synchronization stands out as a critical 
issue, especially given the proliferation and increasing complexity of such systems. 
 In real-world applications, synchronization of robotic systems is crucial across 
various industries. For instance, in manufacturing assembly lines, synchronized 
robots ensure smooth production flow and maximize throughput. Similarly, in 
warehouse logistics, coordinated robotic systems optimize order fulfillment times 
and enhance overall productivity. Furthermore, collaborative robotics scenarios in 
construction projects benefit from synchronized robotic systems for tasks like 
concrete pouring and steel beam placement, improving efficiency and minimizing 
delays. A common challenge in this realm involves effectively controlling robots constrained 
by a shared object. This paper proposes a straightforward methodology to address 
such problems through the Udwadia-Kalaba\cite{UdwadiaKalabaApproach} approach.

This proposed methodology gains particular significance within the context of modern 
physics simulation, derivation, and optimization libraries. These tools offer 
substantial computational speed, enabling efficient problem-solving. In this article, 
we leverage MuJoCo\cite{MuJoCo}, Pinocchio\cite{Pinocchio}, and ProxSuite\cite{CvxPy} 
for simulation, robotic dynamics computation, and optimization, respectively. 
Pinocchio, in particular, emerges as a key tool for computing the dynamics of robotic 
systems. However, its limitation to open-loop physics models poses challenges for 
controlling and synchronizing multiple robots.

Previous solutions have often involved constructing models with pre-existing 
constraints or employing the KKT (Karush-Kuhn-Tucker) approach. However, these 
methods suffer from computational complexity and issues with constraint 
prioritization. The proposed methodology combines the strengths of both approaches, 
integrating physical grounding from the former and simplicity of application from 
the latter. Notably, it allows for manual fine-tuning of constraint priorities and 
boasts superior computational speed through auto code generation. 

The implementation of the proposed methodology demonstrates significant advancements 
in the control and synchronization of robotic systems constrained by shared objects. 
Through the integration of robust physics simulation, precise robotic dynamics 
computation, and efficient optimization techniques, the method achieves remarkable 
outcomes across various simulation experiments. By leveraging 
advanced simulation, computation, and optimization techniques, the method has 
improved levels of efficiency, accuracy, and adaptability in robotic operations, 
paving the way for further advancements in automation and robotics technology.

In subsequent chapters, we delve deeper into various aspects of the proposed method. 
Chapter \ref{chap:lr} provides an exhaustive review of recent literature, 
highlighting the existing landscape of solutions and their limitations. 
Chapter \ref{chap:met} elucidates the methodology underlying the proposed approach, 
offering insights into its theoretical underpinnings. Implementation details and 
code snippets are presented in Chapter \ref{chap:impl}, demonstrating the practical 
application of the method. Chapter \ref{chap:eval} undertakes a comparative 
analysis, pitting technique against established methods to gauge its efficacy. 
Finally, Chapter \ref{chap:conclusion} summarizes findings, discusses 
implications, and outlines avenues for future research.
